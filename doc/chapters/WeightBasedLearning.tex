\newpage
\section{Gewichtbasiertes Lernen}
\subsection{Notationen}
\begin{flushleft}

\begin{align*}
x^{(i)} &= \text{Die Features des i-ten Samples} \\
x_{j}^{(i)} &= \text{Der Wert des j-ten Features im i-ten Samples} \\
m &= \text{Die Anzahl Samples} \\
n &= \text{Die Anzahl Features} \\
X &= \text{Datenset} \\
y &= \text{Target Vektor}
\end{align*}

In der Regel ist ein Datenset als Matrix $X$ gegeben mit $\mathbb{R}^{m \times n}$


\begin{center}
	\begin{table}[h]
	\begin{tabular}{|c|c|c|c|c|}
		\hline
		\textbf{Feature 1} & \textbf{Feature 2} & \textbf{...} & \textbf{Feature n} & \textbf{Target} \\ 
		\hline
		$x_{1}^{(1)}$ & $x_{2}^{(1)}$ & ... & $x_{n}^{(1)}$ & Klasse X  \\ 
		\hline
		$x_{1}^{(2)}$ & $x_{2}^{(2)}$ & ... & $x_{n}^{(2)}$ & Klasse Y  \\ 
		\hline
		$x_{1}^{(m)}$ & $x_{2}^{(m)}$  & ... & $x_{n}^{(m)}$  & Klasse Z  \\ 
		\hline
	\end{tabular}
\end{table}
\end{center}


\end{flushleft}

\newpage
\subsection{Perceptron}
\begin{flushleft}

                        
\begin{align*}
w_{i}\cdot x &= \hat{y} \\
d(\hat{y}) &= \begin{cases}
       		1 & \text{wenn } \hat{y} \geq 0, \\
       		0 & \text{sonst.}
    	\end{cases} \\
d(w_{i} \cdot x) &= \begin{cases}
       		1 & \text{wenn } \hat{y} \geq 0, \\
       		0 & \text{sonst.}
    	\end{cases} \\
d(w_{i}^{T}x) &= \begin{cases}
       		1 & \text{wenn } \hat{y} \geq 0, \\
       		0 & \text{sonst.}
    	\end{cases} \\
\end{align*}
                        
                        
\newcommand{\myThresholdFunction}{
\draw[thick] (-2.25em,0em) -- (1.25em,0em) 
			 (-0.5em,1.75em) -- (-0.5em,-1.75em)
(-0.5em,1.25em) -- (0.5em,1.25em)
(-0.5em,-1.25em) -- (-1.5em,-1.25em)
;}
                        
\begin{tikzpicture}[
     % define styles 
     clear/.style={ 
         draw=none,
         fill=none
     },
     net/.style={
         matrix of nodes,
         nodes={ draw, circle, inner sep=10pt },
         nodes in empty cells,
         column sep=2cm,
         row sep=-9pt
     },
     >=latex
]
% define matrix mat to hold nodes
% using net as default style for cells
\matrix[net] (mat)
{
% Define layer headings
|[clear]| \parbox{1.3cm}{\centering Input\\layer} & 
|[clear]| \parbox{1.3cm}{\centering Gewichtete\\Summe} &
|[clear]| \parbox{1.3cm}{\centering Schwellwert\\Funktion} \\
         
$+1$  		& |[clear]| & |[clear]| \\
|[clear]| 	& |[clear]| & |[clear]| \\
$x_{1}$  	& |[clear]| & |[clear]| \\
|[clear]| 	& $\Sigma$  & \myThresholdFunction \\
\vdots  	& |[clear]| & |[clear]| \\
|[clear]| 	& |[clear]| & |[clear]| \\
$x_{n}$  	& |[clear]| & |[clear]| \\
};
\draw[->] (mat-2-1) -- node[above=1mm] {$w_{0}$} (mat-5-2);
\draw[->] (mat-4-1) -- node[above=1mm] {$w_{1}$} (mat-5-2);
\draw[->] (mat-6-1) -- node[above=1mm] {$\vdots$} (mat-5-2);
\draw[->] (mat-8-1) -- node[above=1mm] {$w_{n}$} (mat-5-2);
\draw[->] (mat-5-2) -- node[above=1mm] {$\hat{y}$} (mat-5-3);
\draw[->] (mat-5-3) -- node[right=2em] {$\begin{cases}
       		1 & \text{wenn } \hat{y} \geq 0, \\
       		0 & \text{sonst.}
    	\end{cases}$} +(2cm,0);
\end{tikzpicture}

\end{flushleft}







