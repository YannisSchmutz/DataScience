\newpage
\section{Normal Equation}
\begin{flushleft}

Die Normal Equation ist eine Alternative zum Gradient Descent und rechnet sich wie folgt:

$$ \Theta = (X^{T}X)^{-1}X^{T}y $$

Unterschied zu Gradient Descent:
\begin{table}[h]
	\begin{tabular}{|l|l|}
		\hline
		{\textbf{Gradient Descent}} 		& {\textbf{Normal Equation}} 		\\ \hline
		Alpha muss gewählt werden       & Alpha muss nicht gewählt werden   \\ \hline
		Benötigt viele Iterationen      & Benötigt keine Iterationen        \\ \hline
		$O(kn^{2})$                     & $O(n^{3})$                        \\ \hline
		Auch für grosses n performant   & Langsam für grosses n             \\ \hline
	\end{tabular}
\end{table}
\end{flushleft}
