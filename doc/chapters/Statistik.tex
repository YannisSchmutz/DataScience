\newpage
\section{Statistik}
\label{sec:stat}

\subsection{Begriffe}
\begin{flushleft}
Dieses Kapitel bietet eine grobe �bersicht �ber einige hilfreiche Begriffe der Statistik.

\subsubsection{Lageparameter}
Lageparameter beschreiben die Lage der Stichprobenelemente im Bezug auf die Messskala.
\linebreak

\textbf{Mittelwert} \\
Auch Durchschnitt (oder mean im Englischen) gennant. In der Wahrscheinlichkeitsrechnung spricht man oft vom Erwartungswert.

$$\bar{x}_{arithm} = \frac{1}{n} \sum_{i=1}^{n} x_i$$

\textbf{Median} \\ 
Der Median oder auch Zentralwert genannt, beschreibt den Wert aus der auf-/ absteigend geordneten Stichprobe, der genau in der Mitte liegt.

\begin{equation*}
  \widetilde{x} = \begin{cases}
    x_{\frac{n+1}{2}}, & \text{\textit{n} ungerade}\\
    \frac{1}{2} \left( x_{\frac{n}{2}} + x_{\frac{n}{1}+1} \right), & \text{\textit{n} gerade}.
  \end{cases}
\end{equation*}


\textbf{Quantile} \\
Schwellenwert (engl. percentile) der angibt, dass ein bestimmter prozentualer Wert einer Menge an Werten kleiner ist als das Quantil. 
Das Quantil bei 50\% ist der Median. Weitere spezielle Quantile sind die Quartile, die Quintile, die Dezile und die Perzentile.
\linebreak

\textbf{Modus} \\
Definiert den h�ufigsten Wert, der in der Stichprobe vorkommt.


\subsubsection{Streuungsparameter}
Streuungsparameter beschreiben die Streuung von Werten einer Stichprobe um einen bestimmten Lageparameter. So ergeben sich je nach gew�hlten Lageparameter unterschiedliche Berechungsformen. Diese unterscheiden sich in ihrer Beeinflussung durch Ausreisser. So wird beispielsweise der Median tendenziell weniger von einem einzelnen, sehr hohen Ausreisser beeinflusst als der arithmetische Mittelwert.
\linebreak

\textbf{Spannweite} \\
Die Spannweite (eng. range) gibt den Abstand des gr�ssten gegen�ber dem kleinsten vorkommenden Wert der Stichprobe an. $R = x_{max} - x_{min}$.
\linebreak
Die Spannweite wird stark durch Ausreisser beeinflusst. Dem kann jedoch durch das alternative Verwenden des \textbf{Interquartilsabstands} (engl. interquartile range) entgegengewirkt werden. Dieser berechnet n�mlich die Spannweite zwischen zwei Quantilen. Somit k�nnen Ausreisser ignoriert werden.
\linebreak

\textbf{Varianz} \\
\linebreak






\textbf{Korrelation}
\linebreak


\textbf{Kovarianz}
\linebreak


\textbf{Kausalit�t}
\linebreak


\end{flushleft}
