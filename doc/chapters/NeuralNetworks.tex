\newpage
\section{Neuronale Netzwerke}
\subsection{Representation}
\begin{flushleft}


Der Inputnode $x_{0}$ wird nicht immer eingezeichnet und repräsentiert den Bias-Node.

Weiter gilt:

$\begin{aligned}
 \alpha_{i}^{(j)} &= \text{Aktivierung der Einheit i im Layer j}
\end{aligned}$

$\begin{aligned}
 \Theta^{(j)} &= \text{Gewichtsmatrix welche Layer j auf Layer j+1 mapt}
\end{aligned}$

\begin{tikzpicture}[
     % define styles 
     clear/.style={ 
         draw=none,
         fill=none
     },
     net/.style={
         matrix of nodes,
         nodes={ draw, circle, inner sep=10pt },
         nodes in empty cells,
         column sep=2cm,
         row sep=-9pt
     },
     >=latex
]
% define matrix mat to hold nodes
% using net as default style for cells
\matrix[net] (mat)
{
% Define layer headings
|[clear]| \parbox{1.3cm}{\centering Input\\layer} 
    & |[clear]| \parbox{1.3cm}{\centering Hidden\\layer} 
    & |[clear]| \parbox{1.3cm}{\centering Output\\layer} \\
         
$x_{0}$  & |[clear]|        & |[clear]| \\
|[clear]|         & $\alpha_{1}^{(2)}$ & |[clear]| \\
$x_{1}$  & |[clear]|        & |[clear]| \\
|[clear]|         & |[clear]|        & |[clear]| \phantom{$a_{0}^{0}$} \\
$x_{2}$  & $\alpha_{2}^{(2)}$ & $$ \\
|[clear]|         & |[clear]|        & |[clear]|  \phantom{$a_{0}^{0}$} \\
$x_{3}$  & |[clear]|        & |[clear]| \\
|[clear]|         & $\alpha_{3}^{(2)}$ & |[clear]| \\
$x_{4}$  & |[clear]|        & |[clear]| \\ 
};
% left most lines into input layers
\foreach \ai in {2,4,6,8,10}
    \draw[<-] (mat-\ai-1) -- +(-2cm,0);
% lines from a_{i}^{0} to each a_{j}^{1}
\foreach \ai in {2,4,6,8,10} {
    \foreach \aii in {3,6,9}
        \draw[->] (mat-\ai-1) -- (mat-\aii-2);
        }
% lines from a_{i}^{1} to a_{0}^{2}
\foreach \ai in {3,6,9}
  \draw[->] (mat-\ai-2) -- (mat-6-3);
    
% right most line with Output label
\draw[->] (mat-6-3) -- node[above] {$h_{\Theta}(x)$} +(2cm,0);
\end{tikzpicture}




$$ \alpha_{1}^{(2)} = g(\Theta_{10}^{(1)}x_{0} +  \Theta_{11}^{(1)}x_{1} + \Theta_{12}^{(1)}x_{2} +\Theta_{13}^{(1)}x_{3} + \Theta_{14}^{(1)}x_{4})$$

$$ \alpha_{2}^{(2)} = g(\Theta_{20}^{(1)}x_{0} +  \Theta_{21}^{(1)}x_{1} + \Theta_{22}^{(1)}x_{2} +\Theta_{23}^{(1)}x_{3} + \Theta_{24}^{(1)}x_{4})$$

$$ \alpha_{3}^{(2)} = g(\Theta_{30}^{(1)}x_{0} +  \Theta_{31}^{(1)}x_{1} + \Theta_{32}^{(1)}x_{2} +\Theta_{33}^{(1)}x_{3} + \Theta_{34}^{(1)}x_{4})$$


$$ h_{\Theta}(x) = \alpha_{1}^{(3)} = g(\Theta_{10}^{(2)}\alpha_{0}^{(2)} + \Theta_{11}^{(2)}\alpha_{1}^{(2)} + \Theta_{12}^{(2)}\alpha_{2}^{(2)} + \Theta_{13}^{(2)}\alpha_{3}^{(2)}) $$


Existiert ein neuronales Netz mit $s_{j}$ Einheiten im Layer $j$, $s_{j+1}$ Einheiten im Layer $j+1$, dann hat $\Theta^{(j)}$ die Dimension $s_{j+1} \times (s_{j} + 1)$.

\end{flushleft}


\subsection{Logik Beispiel}

\subsubsection{AND}
\begin{flushleft}


Ein AND-Gatter kann wie folgt erstellt werden:

\begin{tikzpicture}[
     % define styles 
     clear/.style={ 
         draw=none,
         fill=none
     },
     net/.style={
         matrix of nodes,
         nodes={ draw, circle, inner sep=10pt },
         nodes in empty cells,
         column sep=2cm,
         row sep=-9pt
     },
     >=latex
]
% define matrix mat to hold nodes
% using net as default style for cells
\matrix[net] (mat)
{
% Define layer headings
|[clear]| \parbox{1.3cm}{\centering Input\\layer} & |[clear]| \parbox{1.3cm}{\centering Output\\layer} \\
         
$+1$  		& |[clear]| \\
|[clear]| 	& |[clear]| \\
$x_{1}$  	& |[clear]| \\
|[clear]| 	& $$ \\
$x_{2}$  	& |[clear]| \\
};
\draw[->] (mat-2-1) -- node[above=1mm] {-30} (mat-5-2);
\draw[->] (mat-4-1) -- node[above=1mm] {20} (mat-5-2);
\draw[->] (mat-6-1) -- node[above=1mm] {20} (mat-5-2);
\draw[->] (mat-5-2) -- node[above] {$h_{\Theta}(x)$} +(2cm,0);
\end{tikzpicture}

$$ h_{\Theta}(x) = g(-30 + 20x_{1} + 20x_{2}) $$


\begin{center}
\begin{tabular}{ c c|r } 

 $x_{1}$ & $x_{2}$ & $h_{\Theta}(x)$ \\ 
  \hline
 0 & 0 & $g(-30) \approx 0$ \\ 
 0 & 1 & $g(-10) \approx 0$ \\ 
 1 & 0 & $g(-10) \approx 0$ \\ 
 1 & 1 & $g(10) \approx 1$ \\ 

\end{tabular}
\end{center}
\end{flushleft}


\subsubsection{OR}
\begin{flushleft}


Ein OR-Gatter kann wie folgt erstellt werden:

\begin{tikzpicture}[
     % define styles 
     clear/.style={ 
         draw=none,
         fill=none
     },
     net/.style={
         matrix of nodes,
         nodes={ draw, circle, inner sep=10pt },
         nodes in empty cells,
         column sep=2cm,
         row sep=-9pt
     },
     >=latex
]
% define matrix mat to hold nodes
% using net as default style for cells
\matrix[net] (mat)
{
% Define layer headings
|[clear]| \parbox{1.3cm}{\centering Input\\layer} & |[clear]| \parbox{1.3cm}{\centering Output\\layer} \\
         
$+1$  		& |[clear]| \\
|[clear]| 	& |[clear]| \\
$x_{1}$  	& |[clear]| \\
|[clear]| 	& $$ \\
$x_{2}$  	& |[clear]| \\
};
\draw[->] (mat-2-1) -- node[above=1mm] {-10} (mat-5-2);
\draw[->] (mat-4-1) -- node[above=1mm] {20} (mat-5-2);
\draw[->] (mat-6-1) -- node[above=1mm] {20} (mat-5-2);
\draw[->] (mat-5-2) -- node[above] {$h_{\Theta}(x)$} +(2cm,0);
\end{tikzpicture}

$$ h_{\Theta}(x) = g(-10 + 20x_{1} + 20x_{2}) $$


\begin{center}
\begin{tabular}{ c c|r } 

 $x_{1}$ & $x_{2}$ & $h_{\Theta}(x)$ \\ 
  \hline
 0 & 0 & $g(-10) \approx 0$ \\ 
 0 & 1 & $g(10) \approx 1$ \\ 
 1 & 0 & $g(10) \approx 1$ \\ 
 1 & 1 & $g(30) \approx 1$ \\ 

\end{tabular}
\end{center}
\end{flushleft}


\subsubsection{NOT}
\begin{flushleft}


Ein NOT-Gatter kann wie folgt erstellt werden:

\begin{tikzpicture}[
     % define styles 
     clear/.style={ 
         draw=none,
         fill=none
     },
     net/.style={
         matrix of nodes,
         nodes={ draw, circle, inner sep=10pt },
         nodes in empty cells,
         column sep=2cm,
         row sep=-9pt
     },
     >=latex
]
% define matrix mat to hold nodes
% using net as default style for cells
\matrix[net] (mat)
{
% Define layer headings
|[clear]| \parbox{1.3cm}{\centering Input\\layer} & |[clear]| \parbox{1.3cm}{\centering Output\\layer} \\
         
$+1$  		& |[clear]| \\
|[clear]| 	& $$ \\
$x_{1}$  	& |[clear]| \\
};
\draw[->] (mat-2-1) -- node[above=1mm] {10} (mat-3-2);
\draw[->] (mat-4-1) -- node[above=1mm] {-20} (mat-3-2);
\draw[->] (mat-3-2) -- node[above] {$h_{\Theta}(x)$} +(2cm,0);
\end{tikzpicture}

$$ h_{\Theta}(x) = g(10 - 20x_{1}) $$


\begin{center}
\begin{tabular}{ c|r } 

 $x_{1}$ & $h_{\Theta}(x)$ \\ 
  \hline
 0 & $g(10) \approx 1$ \\ 
 1 & $g(-10) \approx 0$ \\ 

\end{tabular}
\end{center}
\end{flushleft}



\subsubsection{XNOR}
\begin{flushleft}


Durch Kombinationen von einzelnen NN können komplexere Gatter erstellt werden. Wie bspw. das XNOR Gatter.

\begin{tikzpicture}[
     % define styles 
     clear/.style={ 
         draw=none,
         fill=none
     },
     net/.style={
         matrix of nodes,
         nodes={ draw, circle, inner sep=10pt },
         nodes in empty cells,
         column sep=2cm,
         row sep=-9pt
     },
     >=latex
]
% define matrix mat to hold nodes
% using net as default style for cells
\matrix[net] (mat)
{
% Define layer headings
|[clear]| \parbox{1.3cm}{\centering Input\\layer} 
	& |[clear]| \parbox{1.3cm}{\centering Hidden\\layer} 
	& |[clear]| \parbox{1.3cm}{\centering Output\\layer} \\
         
$+1$  		& $+1$ 					&	|[clear]|\\
|[clear]|	& |[clear]|				&	|[clear]| \\
$x_{1}$  	& $\alpha_{1}^{(2)}$  	&  	$\alpha_{1}^{(3)}$\\
|[clear]|	& |[clear]|				&	|[clear]| \\
$x_{2}$  	& $\alpha_{1}^{(2)}$ 	&  	|[clear]|\\
};
% +1
\draw[->] (mat-2-1) -- node[above=0mm] {-30} (mat-4-2);
\draw[->] (mat-2-1) -- node[above=0mm] {10} (mat-6-2);
%  x1
\draw[->] (mat-4-1) -- node[below=0mm] {20} (mat-4-2);
\draw[->] (mat-4-1) -- node[above=0mm] {-20} (mat-6-2);
%  x2
\draw[->] (mat-6-1) -- node[below=0mm] {20} (mat-4-2);
\draw[->] (mat-6-1) -- node[below=0mm] {-10} (mat-6-2);

%
\draw[->] (mat-2-2) -- node[above=0mm] {-10} (mat-4-3);
\draw[->] (mat-4-2) -- node[above=0mm] {20} (mat-4-3);
\draw[->] (mat-6-2) -- node[above=0mm] {20} (mat-4-3);

\draw[->] (mat-4-3) -- node[above] {$h_{\Theta}(x)$} +(2cm,0);

\end{tikzpicture}


\begin{center}
\begin{tabular}{ c c|c c|r } 

 $x_{1}$ & $x_{2}$ & $\alpha_{1}^{(2)}$ & $\alpha_{2}^{(2)}$ & $h_{\Theta}(x)$ \\ 
  \hline
 0 & 0 & 0 & 1 & 1 \\ 
 0 & 1 & 0 & 0 & 0 \\ 
 1 & 0 & 0 & 0 & 0 \\ 
 1 & 1 & 1 & 0 & 1 \\ 

\end{tabular}
\end{center}
\end{flushleft}


