\documentclass{article}

% PACKAGES  ============================================================
\usepackage{lipsum}		% Used to generate dummy-text (\lipsum[1])
\usepackage[margin=2.54cm,includefoot]{geometry}		% Used to control margins
\usepackage{scrextend}	% To be able to use \begin{addmargin}

\usepackage{graphicx} % allows to import images
\usepackage{float}	% allows for control of float positions
\usepackage{tikz} 	% Used to create trees, neural networks, 
\usetikzlibrary{matrix,chains,positioning,decorations.pathreplacing,arrows} % For neural networks
\tikzset{marrow/.style={midway,red,sloped,fill, minimum height=3cm, single arrow, single arrow
    head extend=.5cm, single arrow head indent=.25cm,xscale=0.3,yscale=0.15,
    allow upside down},
    node box/.style={white, draw=black, text=black, rectangle, rounded corners},
    } % Arrows and boxes in random forest

\usepackage{pgfplots}	% to create plots

\usepackage{forest} % Used to create trees
\usetikzlibrary{fit,shapes.arrows,positioning}


\usepackage{amsmath}	 % to use cases in equations and vectors

\usepackage[hidelinks]{hyperref}	% allows for clickable references

\usepackage[numbers,sort&compress]{natbib} 	% sorts the cites in increasing order automatically when referenced and compresses successive references

\usepackage[utf8]{inputenc}		% Be able to use umlaute

\usepackage[ngerman]{babel}

\usepackage{fancyhdr}	% Used for Header and Footer stuff

\usepackage{xfrac}	% allows for slanted fractions 

\usepackage{pgfplots}	% Used for function-plots

\usepackage{amssymb} % Use commands like \mathbb
\usepackage{amsmath}
\usepackage{mathtools} % be able to use :=

\usepackage{algorithm}		% Wirte pseudocode algorithms
%\usepackage{algorithmic}
\usepackage{algpseudocode}

\usepackage{subfig} % Figures side by side
%\usepackage[table]  % Used for tables

% ============================================================

% HEADER AND FOOTER STUFF ============================================================
\pagestyle{fancy}
\fancyhead{}	% clears header
\fancyfoot{}	% clears footer
\fancyfoot[R]{\thepage}	% sets position right
\renewcommand{\headrulewidth}{0pt}	% removes header line by setting it to zero
\renewcommand{\footrulewidth}{1pt}		% add footer line by setting it to one
% ============================================================

% DEFINE LIST ITEM BULLETS ============================================================
\renewcommand{\labelitemi}{$\bullet$}	% first list item
\renewcommand{\labelitemii}{$\circ$}	% one-indented item
\renewcommand{\labelitemiii}{$\diamond$}	% twice-indented item
% ============================================================

% BE ABLE TO USE ROW AND COLUMN VECTORS INLINE ============================================================
\newcommand{\icol}[1]{% inline column vector
  \left(\begin{smallmatrix}#1\end{smallmatrix}\right)%
}

\newcommand{\irow}[1]{% inline row vector
  \begin{smallmatrix}(#1)\end{smallmatrix}%
}
% ============================================================

% R Commands =================================================
% Be able to reference sections with number
\newcommand{\secref}[1]{\autoref{#1}. \nameref{#1}}
% ============================================================


% TODO: Diese beiden sollten jeweils das \begin{flushleft} erstzen. 
% Jedoch ist der Text dann im Blocksatz, wenn flushleft fehlt...
%Einrücken von Absätzen deaktivieren
\setlength{\parindent}{0pt}
%Zeilenabstand bei Abstätzen
\usepackage{parskip}

% ##################################################
% Listings (Sourcecode)
% ##################################################

\usepackage{listings}
%\usepackage{xcolor}
\usepackage{color}

%use typewriter font which supports bold characters
\usepackage{beramono}

\definecolor{codegreen}{rgb}{0,0.6,0}
\definecolor{codegray}{rgb}{0.5,0.5,0.5}
\definecolor{codepurple}{rgb}{0.5,0,0.33}
\definecolor{codepurblue}{rgb}{0.16,0.0,1.0}
\definecolor{backcolour}{rgb}{0.95,0.95,0.92}


% TODO: Set Python colors
\lstdefinestyle{codestyle}{
    backgroundcolor=\color{backcolour},   
    commentstyle=\color{codegreen},
    keywordstyle=\bfseries\color{codepurple},
    numberstyle=\tiny\color{codegray},
    stringstyle=\color{codepurblue},
    basicstyle=\scriptsize\ttfamily,
    breakatwhitespace=false,         
    breaklines=true,                 
    captionpos=b,                    
    keepspaces=true,                 
    numbers=left,                     
    numbersep=5pt,                 
    showspaces=false,                
    showstringspaces=false,
    showtabs=false,                  
    tabsize=2,
    texcl=true,
}

\lstset{style=codestyle}

% ===========================

\begin{document}

% TITLE PAGE ============================================================
\begin{titlepage}
	
	\begin{center}
	\line(1,0){330} \\
	[2mm]
	\huge{\bfseries Data Science Zusammenfassung} \\
	[2mm]
	\line(1,0){320} \\
	[1,5cm]
	\textsc{\LARGE By Yannis Schmutz} \\
	[0.75cm]
	\textsc{\large todo} \\
	
	\end{center}
	
\end{titlepage}
% ============================================================

% PREFACE STUFF ============================================================
\pagenumbering{roman}		% sets the page numbering to roman for the preface etc.
\section*{Zusammenfassung}	% Adds a section without a number in front
\addcontentsline{toc}{section}{\numberline{}Zusammenfassung}	% adds a section without a number in front to the ToC
\cleardoublepage	% Finishes the current page so that the following page will always be odd.
% ****************************************************************

% TABLE OF CONTENTS ============================================================
\renewcommand{\contentsname}{Inhaltsverzeichnis}	% Rename table of contents to the german version
\tableofcontents		% adds table of contents (this needs to be compiled twice sometimes in order to update)
\thispagestyle{empty}	% removes header & footer on this page
\cleardoublepage	% Finishes the current page so that the following page will always be odd.
% ============================================================

% LIST OF FIGURES ============================================================
%\renewcommand{\listfigurename}{Abbildungsverzeichnis}	% renames list of figures
\listoffigures	% generates a list of figures
\addcontentsline{toc}{section}{Abbildungsverzeichnis}	% Adds list of figures to the ToC
\cleardoublepage
% ============================================================

% LIST OF TABLES ============================================================
%\renewcommand{\listtablename}{Tabellenverzeichnis}
\listoftables
\addcontentsline{toc}{section}{Tabellenverzeichnis}
\cleardoublepage
% ============================================================

% START OF REGULAR CHAPTERS ============================================================
\setcounter{page}{1}		% Sets this page to the first one (and not the table of contents)
\pagenumbering{arabic}	% Sets the page numbering back to arabic

\newpage
\section{Einleitung}
\label{sec:stat}


Dieses Dokument beschreibt diverse Themen im Bereich Data Engineering, Machine Learning und Data Science. Der Fokus liegt auf dem behandeltem Stoff des Moduls DENG2 der Berner Fachhochschule. \\

\textbf{Prüfungsrelevant} sind die folgenden Kapitel:

\begin{itemize}
  \item \secref{sec:weight_based_learning}
  \item \secref{sec:continuous_learning}
  \item \secref{sec:logistic_regression_basics}
\end{itemize}

\include{chapters/02_Statistik}
\include{chapters/03_Probabilistik}

% ****************************************************************
\newpage
\section{Feature Scaling}

\includegraphics[scale=0.5]{figures/feature_notation}

\subsection{Mean normalization}

$$ x_{j} = \frac{x_{j}^{i} - \mu_{j}}{max(x_{j}) - min(x_{j})} \qquad\forall x^{i} \in x_{j}$$

% ****************************************************************
\include{chapters/05_MachineLearning}
\include{chapters/06_WeightBasedLearning}
\newpage
\section{Continuous Learning}
\label{sec:continuous_learning}

Beim obigen Beispiel (Perceptron) wurde der Error berechnet indem der Vorhersagewert (Klasse, 1 oder 0) vom erwarteten Wert subtrahiert wurde.

In diesem Kapitel werden Ansätze angeschaut, welche auch schauen \textbf{wie weit weg} die Vorhersage war.

\subsection{Adaline}
%\begin{flushleft}

Adaline steht für \textbf{Adaptive Linear Neuron} und ist ein \textbf{supervised classification Algorithmus}. Adaline funktioniert wie folgt:

\begin{itemize}
  \item Die Elemente des Input-Vektors (Featurevektors) werden jeweils mit einem Gewicht multipliziert und aufsummiert.
  \item Die Summe (z) wird einer \textbf{Aktivierungsfunktion} übergeben, welche den Wert $\hat{y}$ erzeugt.
  \item $\hat{y}$ wird dann verwendet, um den \textbf{Fehler} bzw. das \textbf{Update für die Gewichte} zu berechnen.
  \item $\hat{y}$ wird zudem einer \textbf{Schwellwertfunktion} übergeben, welche die Features einer Klasse zuordnet.
\end{itemize}




\newcommand{\myThresholdFunction}{
\draw[thick] %(-2.25em,0em) -- (1.25em,0em) 
			 (-0.5em,1.25em) -- (-0.5em,-1.25em)
(-0.5em,1.25em) -- (0.5em,1.25em)
(-0.5em,-1.25em) -- (-1.5em,-1.25em)
;}


\begin{figure}[H]
\centering
\label{fig:perceptron}
\begin{tikzpicture}[
     % define styles 
     clear/.style={ 
         draw=none,
         fill=none
     },
     net/.style={
         matrix of nodes,
         nodes={ draw, circle, inner sep=10pt },
         nodes in empty cells,
         column sep=1.5cm,
         row sep=-9pt
     },
     >=latex
]
% define matrix mat to hold nodes
% using net as default style for cells
\matrix[net] (mat)
{
% Define layer headings
|[clear]| \parbox{1.3cm}{\centering Input\\layer} & 
|[clear]| \parbox{1.3cm}{\centering Gewichtete\\Summe} &
|[clear]| \parbox{1.3cm}{\centering Aktivierungs\\funktion} &
|[clear]| \parbox{1.3cm}{\centering Schwellwert\\Funktion} \\
         
$+1$  		& |[clear]| & Error     & |[clear]| \\
|[clear]| 	& |[clear]| & |[clear]| & |[clear]| \\
$x_{1}$  	& |[clear]| & |[clear]| & |[clear]| \\
|[clear]| 	& $\Sigma$  & $\phi$ & \myThresholdFunction \\
\vdots  	& |[clear]| & |[clear]| & |[clear]| \\
|[clear]| 	& |[clear]| & |[clear]| & |[clear]| \\
$x_{n}$  	& |[clear]| & |[clear]| & |[clear]| \\
};
\draw[->] (mat-2-1) -- node[above=1mm] {$w_{0}$} (mat-5-2);
\draw[->] (mat-4-1) -- node[above=1mm] {$w_{1}$} (mat-5-2);
\draw[->] (mat-6-1) -- node[above=1mm] {$\vdots$} (mat-5-2);
\draw[->] (mat-8-1) -- node[above=1mm] {$w_{n}$} (mat-5-2);
\draw[->] (mat-5-2) -- node[above=1mm] {$z$} (mat-5-3);
\draw[->] (mat-5-3) -- node[above=1mm] {$\hat{y}$} (mat-5-4);
\draw[->] (mat-5-3) -- node[above=1mm] {$$} (mat-2-3);
\draw[->] (mat-2-3) -- node[above=1mm] {Update Gewichte} (-2.5cm, 1cm);

\draw[->] (mat-5-4) -- node[right=2em] {$\begin{cases}
       		1 & \text{wenn } \hat{y} \geq 0, \\
       		0 & \text{sonst.}
    	\end{cases}$} +(2cm,0);
\end{tikzpicture}
\caption{Adaline als Model}
\label{fig:adaline_model}
\end{figure}


\newpage
Hier eine vereinfachte Beschreibung des Adaline-Models.

\textbf{Training} \\
Füge Bias-Gewicht (1) an Stelle Feature-0 hinzu.
Bestimme $z$:
\begin{align*}
	z &= w^{T} \times x \\
	z &= \sum_{i=0}^{n} w_{i}x_{i}  \\
\end{align*}


Einfachheitshalber nutzen wir für die \textbf{Aktivierungsfunktion} die Identitätsfunktion:
 $$\phi(z) = z$$
 
 Rechne nach jeder \textbf{Epoche} den Error aus und aktualisiere dementsprechend die Gewichte.



\textbf{Prediction} \\
Nun benötigen wir nur noch eine \textbf{Schwellwertfunktion} (decision function), die die vorhergesagte Klasse bestimmt.

$$ d(\hat{y}) =
\begin{cases}
	1 & \text{wenn } \hat{y} \geq 0, \\
	0 & \text{sonst.}
\end{cases}
$$


% ============================================
\newpage
\subsubsection{Adaline vs Perceptron}

Ähnlich wie der Perceptron ist Adaline ein Einzellayer neuronales Netzwerk. Der Hauptunterschied liegt auf der Aktivierungsfunktion phi $\phi(z)$


\begin{itemize}
  \item Der \textbf{Perceptron} aktualisiert die Gewichte nur, wenn eine falsche Vorhersage getroffen wurde. Zudem wird die Error-Funktion erst nach der \textbf{Schwellwertfunktion} aufgerufen. Somit wird ihr stets immer nur eine 0 oder 1 übergeben.
  \item \textbf{Adaline} hingegen aktualisiert die Gewichte basierend auf einer \textbf{stetigen} Funktion (continuous). Der \textbf{Aktivierungsfunktion $\phi$}
 \end{itemize}



Wie in Abbildung \ref{fig:adaline_vs_perceptron} ersichtlich ist, werden die Gewichte bei Adaline \textbf{vor} der Entscheidungsfunktion aktualisiert.

\begin{figure}[h!]
	\includegraphics[scale=0.6]{figures/adaline_vs_perceptron}
	\caption{Adaline vs Perceptron}
	\label{fig:adaline_vs_perceptron}
\end{figure}

Im Gegensatz zum \textbf{binären Lernansatz} des Perceptron basieren viele supervised learning Algorithmen auf einer sogenannten \textbf{objective learning function}.


\newpage
\subsubsection{Objective Function}

Die Objective Function (Zielfunktion) ist mathematisch eine \textbf{Optimierung}.

Das Ziel hierbei ist es \textbf{optimale Parameter} zu finden. Optimal bedeutet den Output der Funktion entweder zu \textbf{maximieren} oder zu \textbf{minimieren}. 


Bei Machine Learning entsprechen die Parameter den \textbf{Gewichten}.


Eine Objective Funktion berechnet einen Output basierend auf den Eingaben:

\begin{itemize}
  \item Vorhersage (prediction)
  \item Eigentlicher Wert (labelled value)
\end{itemize}

Hierbei soll deren Differenz möglichst klein werden. Somit ist die Vorhersage sehr nahe am eigentlichen Wert. Ein Beispiel dafür zeigt Abbildung \ref{fig:objective_minimum}. 


\begin{figure}[h!]
	\includegraphics[scale=0.4]{figures/objective_minimum}
	\caption{Beispiel Zielfunktion}
	\label{fig:objective_minimum}
\end{figure}



Objective Funktion ist ein sehr genereller Term im Bereich ML.
Meistens wollen wir den Output der Objective Funktion \textbf{minimieren}. In diesem Fall sprechen wir von einer \textbf{Cost Function} oder \textbf{Loss Function}.


Wollen wir den Output \textbf{maximieren} so sprechen wir von einer \textbf{Likelihood Maximization} Funktion.



 
\newpage
\subsubsection{Objective Function in Adaline}

Adaline verwendet die Cost Function \textbf{Sum of Squared Errors (SSE)}.

SSE summiert alle quadrierten Differenzen zwischen Vorhersage und effektivem Wert auf.

$$ SSE = \frac{1}{2} \sum_{i=1}^{m}(y^{(i)} - \hat{y}^{(i)})^{2} $$

Hier eine detailliertere Darstellung von $\hat{y}$

$$ SSE = \frac{1}{2} \sum_{i=1}^{m}(y^{(i)} - \phi(z^{(i)}))^{2} $$


Wichtig zu wissen:
\begin{itemize}
  \item Die Funktion SSE ist \textbf{differenzierbar (ableitbar)}. Daher kann für jeden Punkt die Steigung berechnet werden.
  \item Sie hat ein \textbf{globales Minimum}.
\end{itemize}


Diese beiden Punkte sind notwendig für \textbf{Optimierungsalgorithmen}. Diese helfen dabei den Wert der Gewichte zu bestimmen damit die Cost Function möglichst klein wird. (Bsp. Gradient Descent)


\subsubsection{Adaline Learning Rule}

Die Gewicht-Updates werden anhand von \textbf{allen Samples} berechnet (Summe von 1 bin m). Daher werden die Gewichte auch immer erst nach einem vollen Trainingsdurchlauf aktualisiert und nicht nach jedem Sample.

Diese Art von Aktualisierung wird \textbf{batch gradient descent} genannt.

\begin{align*}
	\Delta w &= \eta \sum_{i=1}^{m} (y^{(i)} - \phi(z^{(i)}))^{2} * x^{(i)} \\
	w &= w + \Delta w \\
\end{align*}

\begin{align*}
	w &= \text{Gewichtsvektor} \\
	\Delta w &= \text{Gewichtsupdate (Vektor)} \\
	\eta &= \text{Learning Rate} \\
	m &= \text{Anzahl Samples}	\\
	y^{(i)} &= \text{Effektiver Zielwert (Label)} \\
	\phi &= \text{Aktivierungsfunktion} \\
	z^{(i)} &= \text{Summe der Input-Gewicht-Multiplikationen} \\
	x^{(i)} &= \text{Feature-Vektor des Samples i} \\
\end{align*}


\subsubsection{Adaline Optimierung mit Gradient Descent}

Für Adaline kann \textbf{Gradient Descent Algorithmus} zur Justierung verwendet werden.

Die Gewichtsupdates können nun so berechnet werden:

$$ \Delta w = - \eta \nabla J(w) $$

\begin{align*}
	J &= \text{Kostenfunktion (bspw. SSE)} \\
	\nabla &= \text{Symbol für Gradient (Steigung)} \\
	w &= \text{Gewichtsvektor} \\
\end{align*}


Wir berechnen also jeweils die \textbf{Steigung der cost function} am Punkt $P(w, J(w))$.

Der Gradient Descent wird in Kapitel \ref{sec:gradient_descent_basics} genauer erklärt.

% ==================================
\subsection{Gradient Descent Basics}
\label{sec:gradient_descent_basics}

Dieses Kapitel behandelt den Stoff zu Gradient Descent, welcher im Rahmen der Vorlesung behandelt wurde. Vertiefte Informationen zu Gradient descent sind Kapitel \ref{sec:gradient_descent} zu entnehmen.


\begin{itemize}
	\item Die \textbf{optimalen Gewichte} für eine \textbf{objective function} zu finden ist sehr rechenintensiv.
	\item Umso mehr Features vorhanden sind, desto mehr Gewichte hat unser Model.
	\item Daher auch mehr Kombinationen, wie die Gewichte gewählt werden könnten.
\end{itemize}

Optimisierungsalgorithmen (bspw. Gradient Descent) helfen dabei eine effiziente Lösung für die optimalen Gewichte zu finden.


\newpage
\subsubsection{Funktionsweise}

Abbildung \ref{fig:gradient_descent_simple} stellt eine vereinfachte Funktionsweise des GD dar.

\begin{itemize}
	\item Auf der y-Achse sind die Kosten (cost function)
	\item Auf der x-Achse sind die Gewichte. In diesem Beispiel haben wir nur ein Gewicht. Eigentlich wäre die Funktion aber multidimensional.
	\item Der schwarze Punkt zeigt die Ausgangslage bei zufällig initialisierten Gewichten.
	\item Der GD rechnet nun jeweils für die aktuelle Position $P(w, J(w))$ die Steigung aus.
	\item Ist die Steigung positiv müssen wir nach links gehen, um näher ans optimale Minimum zu gelangen. Also das Gewicht reduzieren.
	\item Ist die Steigung negativ, so gehen wir nach rechts indem wir das Gewicht erhöhen.
	\item Die Grösse der Schritte, die wir nach Links oder Rechts gehen wird von der \textbf{learning rate $\eta$} beeinflusst.
\end{itemize}

\begin{figure}[h!]
	\includegraphics[scale=0.5]{figures/gradient_descent_simple}
	\caption{Gradient Descent Funktionsweise}
	\label{fig:gradient_descent_simple}
\end{figure}

Abbildung \ref{fig:gd_simple_multi} zeigt wie der GD bei mehreren Gewichten aussehen könnte.


\newpage
\subsubsection{Verhalten des GD}

Eine \textbf{optimale cost function ist konvex}. Also wie eine Schüssel geformt. Somit hat die Funktion nur \textbf{ein} Minimum. (Abbildung \ref{fig:gradient_descent_simple})


In der Praxis sehen die Funktionen aber meist wie
bspw. Abbildung \ref{fig:gd_simple_multi} aus. Diese Funktion hat ein \textbf{globales Minimum} und mehrere \textbf{lokale}. Das globale ist der optimale Wert unserer cost function. Doch nicht immer vom GD erreichbar.

\begin{figure}[h!]
	\includegraphics[scale=0.6]{figures/gd_simple_multi}
	\caption{Gradient Descent multidimensional}
	\label{fig:gd_simple_multi}
\end{figure}


\begin{itemize}
	\item Wird die \textbf{learning rate zu hoch} gewählt, ist es möglich, dass der GD stets \textbf{über} dem Minimum rüber springt.
	\item Ist sie \textbf{zu tief}, so könnte der GD in einem \textbf{lokalen Minumum} festsitzen.
\end{itemize}

\begin{figure}
    \centering
    \subfloat[]{{\includegraphics[scale=0.5]{figures/gd_behaviour} }}%
    \qquad
    \subfloat[]{{\includegraphics[scale=0.5]{figures/gd_learning_rate} }}%
    \caption{GD learning rate}%
    \label{fig:gd_learning_rate}%
\end{figure}


\newpage
\subsubsection{Arten von GD}

Es gibt verschiedene Möglichkeiten den Gradient Descent zu berechnen.

\textbf{Batch Gradient Descent (BGD)} \\


\begin{itemize}
	\item Verwendet \textbf{alle} Samples der Trainingsdaten \textbf{für eine einzige} Gewichtsaktualisierung.
	\item Für jedes Gewicht (bzw. Feature) wird der Error berechnet, indem der Durchschnittswert über alle Samples verwendet wird.
	\item Outliers haben weniger Einfluss auf die Gewichtsanpassung.
\end{itemize}

Dies führt zu einem stabilen Pfad hin zur Konvergenz gegen das Minimum. \\

Nachteile des BGD:

\begin{itemize}
	\item Schwierig zu \textbf{Skalieren}
	\begin{itemize}
		\item Alle Trainingsdaten (Matrix X) muss in RAM passen.
		\item Pro eine Gewichtsanpassung muss das ganze Trainingset einmal durchlaufen werden. Benötigt viel Epochen.
		\item Heutzutage nicht mehr ganz so tragisch.
	\end{itemize}
	\item Resultierendes Model hat Schwierigkeiten zu \textbf{generalisieren}
	\begin{itemize}
		\item Model leidet oftmals an \textbf{overfitting}
		\item Hauptgrund, um BGD nicht zu verwenden.
	\end{itemize}
\end{itemize}




\textbf{Stochastic Gradient Descent (SGD)} \\

Der stochastischen GD aktualisiert die Gewichte nach \textbf{jedem Schritt (nach jedem einzelnen Sample)}. Dies ist hilfreich für \textbf{grössere Mengen an Trainingsdaten}, da weniger Epochen ausgeführt werden müssen. SGD ist also 

\begin{itemize}
	\item Ist effizienter zu \textbf{Skalieren}
	\item Kleinere Tendenz für \textbf{overfitting} im Vergleich zu BGD.
\end{itemize}


Die Gewichte werden nach jedem Schritt wie folgt berechnet:
$$ w = w - \eta \nabla_{w} J(x^{(i)}, y^{(i)}, w) $$


Nachteile des SGD:

\begin{itemize}
	\item Benötigt viele Trainingsschritte
	\item Ausreisser können den SGD in eine falsche Richtung führen.
\end{itemize}


\newpage
\textbf{Mini-Batch Gradient Descent (MBGD)} \\

Wird oft verwendet. Ist ein \textbf{Kompromiss} zwischen BGD und SGD.
Das Gewichtsupdate wird anhand von $b$ vielen Samples berechnet.

$$ w = w - \eta \nabla_{w} J(x^{(i:i+b)}, y^{(i:i+b)}, w) $$ \\


\begin{itemize}
	\item Hilft \textbf{Overfitting / generalization error} zu vermeiden
	\begin{itemize}
		\item Die Batchgrösse $b$ muss wesentlich kleiner sein als die Anzahl Samples $m$.
		\item Dadurch wird dem Lernprozess \textbf{Noise} beigefügt.
	\end{itemize}
	\item Ist \textbf{schneller} als SGD
	\begin{itemize}
		\item Berechnungen erfolgen auf einer Matrix und nicht auf einzelnen Werten.
		\item Dies erlaubt effiziente Berechnungen mittels \textbf{Vektorisierung}.
	\end{itemize}
\end{itemize}


\textbf{Weitere Varianten} \\

\begin{itemize}
	\item AdaGrand (Adaptive Gradient Descent)
	\begin{itemize}
		\item Verschiedene learning rates für jedes Feature
		\item Nicht alle Features haben die gleichen \textbf{Seltenheit} (sparsity) und Werteverteilungen.
		\item Eine höhere learning rate für seltene (sparse) Features steigert deren Beitrag.
		\item Gut geeignet für NLP.
	\end{itemize}
	\item RMSprop
	\begin{itemize}
		\item Learning rate nimmt im Verlauf der Zeit ab.
		\item Umso kleiner die Steigung wird, desto kleiner wird die learning rate.
	\end{itemize}
\end{itemize}






% ****************************************************************
\newpage
\section{Gradient Descent}
\label{sec:gradient_descent}

\begin{flushleft}

Gradient Descent ist ein Optimierungsalgorithmus, um ein lokales Minimum einer Funktion zu finden.
Gegeben sei eine Kosten-Funktion $J(\Theta_{0}, \Theta_{1})$, gesucht ist das Minimum der Funktion $min_{\Theta_{0}, \Theta_{1}} J(\Theta_{0}, \Theta_{1})$, indem die Parameter $\Theta_{0}$ und $\Theta_{0}$ laufend ein wenig verändert werden.

\includegraphics[scale=0.6]{figures/gradient_descent}

Wiederholen bis zur Konvergenz:
$$\Theta_{i} := \Theta_{i} - \alpha\frac{\partial}{\partial \Theta_{i}} J(\Theta_{0}, \Theta_{1}) $$

$$ \alpha: Learning Rate $$
$$ i=0, i=1 $$

\subsubsection{GD für Linear Regression}
Model Funktion:
$$h_{\Theta} = \Theta_{0} + \Theta_{1}x$$
Cost-Function:
$$ J(\Theta_{0}, \Theta_{1}) = \frac{1}{2m} \sum_{i=1}^{m}(h_{\Theta}(x^{(i)})-y^{(i)})^{2} $$

Für lineare Regression hat die Kostenfunktion stets nur ein lokales (sprich globales) Minimum.
\linebreak
\includegraphics[scale=0.6]{figures/cost_function_linear_regression}
\linebreak
Der Gradient Descent Algorithmus für eine simple Lineare Regression rechnet sich wie folgt:
\linebreak


\begin{algorithm}
\caption{Calculate $\text{min}_{\Theta} J(\Theta)$}
\begin{algorithmic} 
\Repeat 
\State $ \Theta_{0} := \Theta_{0} - \alpha \frac{1}{m} \sum_{i=1}^{m}(h_{\Theta}(x^{(i)}) - y^{(i)}) $

\State $ \Theta_{1} := \Theta_{1} - \alpha \frac{1}{m} \sum_{i=1}^{m}(h_{\Theta}(x^{(i)}) - y^{(i)})*x^{(i)} $
\Comment{simultaneously update all $\Theta_{j}$}
\Until{$J(\Theta_{0}, \Theta_{1})$ converges}
\end{algorithmic}
\end{algorithm}


\end{flushleft}
% ****************************************************************



\newpage
\section{Normal Equation}
\begin{flushleft}

Die Normal Equation ist eine Alternative zum Gradient Descent und rechnet sich wie folgt:

$$ \Theta = (X^{T}X)^{-1}X^{T}y $$

Unterschied zu Gradient Descent:
\begin{table}[h]
	\begin{tabular}{|l|l|}
		\hline
		{\textbf{Gradient Descent}} 		& {\textbf{Normal Equation}} 		\\ \hline
		Alpha muss gewählt werden       & Alpha muss nicht gewählt werden   \\ \hline
		Benötigt viele Iterationen      & Benötigt keine Iterationen        \\ \hline
		$O(kn^{2})$                     & $O(n^{3})$                        \\ \hline
		Auch für grosses n performant   & Langsam für grosses n             \\ \hline
	\end{tabular}
\end{table}
\end{flushleft}

\include{chapters/10_LinearRegression}
\newpage
\section{Logistic Regression Basics}
\label{sec:logistic_regression_basics}

Dieses Kapitel beschreibt die fundamentalen Aspekte (prüfungsrelevanter Teil) der logistischen Regression. Die logistic regression ist ein linearer suppervised Klassifikationsalgorithmus. Genauer genommen einen \textbf{binären} Klassifikationsalgorithmus. \\


\subsection{Logistic Regression vs Adaline}

Die grundlegende Architektur von logistic regression ist sehr ähnlich wie bei Adaline. Der \textbf{Hauptunterschied} liegt bei der Aktivierungsfunktion. Die logistic regression verwendet die sogenannte \textbf{Sigmoid} Funktion.

Für Adaline haben wir lediglich die Identitätsfunktion verwendet.


\begin{figure}[h!]
	\includegraphics[scale=0.5]{figures/logistic_regression_vs_adaline}
	\caption{Logistic Regression vs Adaline}
	\label{fig:lr_vs_adaline}
\end{figure}


\newpage
\subsection{Sigmoid Funktion}

Die Sigmoid Funktion nimmt einen Input und transformiert diesen in ein Bereich [0,1].

\begin{tikzpicture}[declare function={sigma(\x)=1/(1+exp(-\x));}]
\begin{axis}%
[
    grid=major,     
    xmin=-8,
    xmax=8,
    axis x line=bottom,
    ytick={0,.5,1},
    ymax=1,
    axis y line=middle,
    samples=100,
    domain=-8:8,
    legend style={at={(1,0.9)}}     
]
    \addplot[blue,mark=none]   (x,{sigma(x)});
    \legend{$g(x)$}
\end{axis}
\end{tikzpicture}


$$ \phi(z) = \frac{1}{1 + e^{-z}} $$

Der Wert von $\phi(z)$ kann auch als \textbf{Wahrscheinlichkeit} interpretiert werden, das ein gewisses Sample zu einer bestimmten Klasse gehört.


$$ \phi(z) = \frac{1}{1 + e^{-z}} = P(y=1 | x;w) $$
Hierbei steht $y=1$ für die binäre Klasse 1.
Somit gilt auch der Zusammenhang zur anderen Klasse 0:
$$ P(y=0|x;w) = 1 - P(y=1|x;w)$$


\subsection{Vorteile logistic regression}

\begin{itemize}
	\item Output der Sigmoid kann als Wahrscheinlichkeit interpretiert werden. Dies gibt uns Informationen über die \textbf{Gewissheit} des Models.
	\item Da logistic regression ein \textbf{lineares Model} ist, ist es einfach zu interpretieren, updaten und ist skalierbar.
\end{itemize}

Diese Vorteile machen die logistische Regression zu einer zuverlässige und sinnvolle Wahl für viele Klassifizierungsprobleme.





\newpage
\section{Logistic Regression}
\subsection{Binary classification}
\begin{flushleft}


Mittels Logistischer Regression lassen sich diskrete Phänomene klassifizieren. Die Funktion des Models ist wie folgt definiert:


$$ h_{\Theta}(x) = g(\Theta^{T}x) $$ 
x ist hierbei lediglich der Featurevektor eines Samples $\icol{x_{1}\\x_{2}\\x_{3}}$. Eine vektorielle Implementation für alle Featuresamples (Matrix) ist weiter unten beschrieben.
$$ g(z) = \frac{1}{1 + e^{-z}} $$
$$ h_{\Theta}(x) = \frac{1}{1 + e^{-\Theta^{T}x}} $$

Die Funktion $g(x)$ ist hierbei die sogenannte Sigmoid (oder auch logistische) Funktion.

\begin{tikzpicture}[declare function={sigma(\x)=1/(1+exp(-\x));}]
\begin{axis}%
[
    grid=major,     
    xmin=-8,
    xmax=8,
    axis x line=bottom,
    ytick={0,.5,1},
    ymax=1,
    axis y line=middle,
    samples=100,
    domain=-8:8,
    legend style={at={(1,0.9)}}     
]
    \addplot[blue,mark=none]   (x,{sigma(x)});
    \legend{$g(x)$}
\end{axis}
\end{tikzpicture}

Der Wert $h_{\Theta}(x)$ wird als Wahrscheinlichkeit verstanden, dass der Output (y) positive ist für ein gegebener Eingabewert (x) parametrisiert mit $\Theta$.
\linebreak

Formal definiert:
$$h_{\Theta}(x) = P(y=1|x;\Theta)$$
Somit gilt:
$$ P(y=0|x;\Theta) = 1 - P(y=1|x;\Theta)$$



Cost-Function:

$$ J(\Theta) = \frac{1}{m}\sum_{i=1}^{m}Cost(h_{\Theta}(x^{(i)}), y^{(i)}) $$


$$ \text{Cost}(h_{\Theta}(x), y) =
    \begin{cases}
      -\text{log}(h_{\Theta}(x)) & \text{for } y=1\\
      -\text{log}(1 - h_{\Theta}(x)) & \text{for } y=0\\
    \end{cases} $$

Im Gegensatz zur linearen Regression wird bei der logistischen die Cost-Funktion je nach Fall (Wert von y) unterschieden, um eine konvexe Funktion zu generieren.
Graphisch dargestellt sieht die Cost-Function wie folgt aus:

\begin{tikzpicture}[declare function={c1(\x)=-log2(x); c0(\x)=-log2(1-x);}]
\begin{axis}%
[
    grid=major,     
    xmin=-0.1,
    xmax=1,
    axis x line=bottom,
    ytick={0,1,2,3,4,5},
    ymax=5,
    axis y line=middle,
    samples=1000,
    domain=0:1,
    legend style={at={(1,0.9)}}     
]
    \addplot[blue,mark=none]   (x,{c1(x)});
    \addplot[red,]   (x,{c0(x)});
    \legend{$\text{cost1}(x)$, $\text{cost0}(x)$}
\end{axis}
\end{tikzpicture}


\begin{algorithm}
\caption{Calculate $\text{min}_{\Theta} J(\Theta)$}
\begin{algorithmic} 
\Repeat 
\State $ \Theta_{j} := \Theta_{j} - \alpha\sum_{i=1}^{m}(h_{\Theta}(x^{(i)}) - y^{(i)})x_{j}^{(i)} $
\Comment{simultaneously update all $\Theta_{j}$}
\Until{$J(\Theta)$ converges}
\end{algorithmic}
\end{algorithm}

Eine Vektor-Implementation dieses Algorithmus sieht so aus:

$$ \Theta := \Theta - \frac{\alpha}{m} X^{T}(g(X\Theta) - \vec{y}) $$

\subsubsection{Regularization}

TODO


% ****************************************
\subsection{Multiclass classification}
\subsubsection{One-vs-all}

Sei $y = {0, 1, ..., n}$ die zu klassifizierenden Klassen. So wird für jede Klasse in y eine logarithmische Regression gebildet. Die entstehende Funktion unterscheidet zwischen der gerade betrachteten Klasse (i) und allen anderen. Daher wird diese Methode auch One-vs-rest genannt.

$$ y \in {0, 1, ..., n} $$

$$ h_{\Theta}^{(0)}(x) = P(y=0|x;\Theta) $$
$$ h_{\Theta}^{(1)}(x) = P(y=1|x;\Theta) $$
$$ \text{...} $$
$$ h_{\Theta}^{(n)}(x) = P(y=n|x;\Theta) $$

$$ \text{prediction} = \text{max}_{i}(h_{\Theta}^{(i)}(x)) $$








\end{flushleft}




\newpage
\section{Decision Tree}

Der Decision Tree ist ein \textbf{supervised learning Algorithmus}. Er kann sowohl für Klassifizierung wie auch für Regression verwendet werden. Wir konzentrieren uns hierbei aber lediglich auf die \textbf{Klassifizierung}. \\

\textbf{Grundlegende Idee:} \\
Basierend auf den Target-Labels und deren Features wird eine \textbf{Serie von Entscheidungen} erstellt. Entsprechend diesen Entscheidungen wird einem neuen (unbekannten) Sample eine Klasse zugewiesen. 

%===
\subsection{Funktionsweise des Decision Trees}

Der Decision Trees besteht, wie der Name schon sagt, aus einer Baum-Datenstruktur. 

Die \textbf{internen Knoten} (internal/branch nodes) beinhalten jeweils eine Entscheidung. Die \textbf{externen Knoten} (external/outer/leaf/ node) beinhalten nur noch Samples einer bestimmten Klasse (Idealfall!).

Während der \textbf{Generierung des Baumes (Lernphase)} werden die Trainingsdaten bei jedem internen Node weiter aufgeteilt.

In Abbildung \ref{fig:dt_simple} sind die internen Knoten Gelb, die externen Knoten Blau dargestellt.

\begin{figure}[h!]
	\includegraphics[scale=0.8]{figures/decision_tree_simple}
	\caption{Decision Tree simples Beispiel}
	\label{fig:dt_simple}
\end{figure}

Um einen Decision Tree erstellen zu können muss erst das Folgende geklärt werden:
\begin{itemize}
	\item Bestimmen wie die Samples aufgeteilt werden
	\begin{itemize}
		\item \textbf{Welche Attribute} (Features) sollen wann zum Splitten verwendet werden?
		\item Wie bestimmen wir den \textbf{besten} Split?
		\item Wie bestimmen wir die \textbf{Bedingung der Aufteilung}? Bspw. wenn Feature $x_{1} < 1$ ist.
		\item Verwenden wir einen \textbf{2er (2-way)} Split oder einen \textbf{multi-way} Split?
	\end{itemize} 
	\item Bestimme wann mit dem Splitten \textbf{aufgehört} wird (Stop-Kriterium).
\end{itemize} 

% ======
\subsubsection{Bestimmung der Aufteilung}

Grundsätzlich wollen wir bei jedem Knoten den höchsten Grad an Aufteilung (bezüglich der Klassen) erreichen. \\

Betrachten wir das Beispiel in Abbildung \ref{fig:dt_split_data}, in welchem anhand von Grösse, Gewicht und Arbeitsstelle das Geschlecht erkennt werden soll. Einfachheitshalber wurden jeweils binäre Werte verwendet.

\begin{figure}[h!]
	\includegraphics[scale=0.8]{figures/decision_tree_ex1_data}
	\caption{Decision Tree Beispiel Aufteilung Daten}
	\label{fig:dt_split_data}
\end{figure}

Nun stellt sich die Frage welches Feature (klein, leicht, Arbeitet in Tech) für die erste Entscheidung des Decision Trees verwendet werden soll. Also bspw. ist die Person klein? \\

Dafür rechnet der DT für jedes Feature (für jede möglich Decision) den \textbf{Information Gain} mit Hilfe der \textbf{Entropy} aus. \\

In Abbildung \ref{fig:dt_split} ist der DT zum obigen Beispiel. Man beachte, dass jeweils die Entropy der Knoten angezeigt wird.


\newpage
\begin{figure}[h!]
	\includegraphics[scale=0.6]{figures/decision_tree_ex1}
	\caption{Decision Tree Beispiel Aufteilung}
	\label{fig:dt_split}
\end{figure}

Die \textbf{Entropy} eines einzelnen Knoten rechnet sich wie folgt:
$$ H(X) = - \sum_{c \in C} p_{c} * log_{2}(p_{c}) $$

\begin{align*}
	H(X) &= \text{Entropy des Datensets X} \\
	C &= \text{Menge der Target-Klassen (bspw. {Mann, Frau})} \\
	p_{c} &= \text{Wahrscheinlichkeit der Klasse c} \\
	      &= \frac{\text{Anzahl Samples der Klasse c}}{\text{Grösse des betrachteten Datensets}}  \\
\end{align*}

Beim ersten Knoten des DT haben wir 8 Samples ($|X| = 8$) und jeweils vier Männer und vier Frauen. Somit:

$$ H(X) = -\left(\frac{4}{8} * log_{2}\left(\frac{4}{8}\right) + \frac{4}{8} * log_{2}\left(\frac{4}{8}\right)\right) = 1 $$


Beim zweiten inneren Knoten des DT haben wir noch 5 Samples ($|X| = 5$) bestehend aus vier Frauen und einem Mann. Somit:

$$ H(X) = -\left(\frac{4}{5} * log_{2}\left(\frac{4}{5}\right) + \frac{1}{5} * log_{2}\left(\frac{1}{5}\right)\right) = 0.722 $$



Der untere Graph stellt die Funktion $H(X)$ für zwei Klassen dar. Die  Entropy kann auch als \textbf{Unordnung / Mix aus Klassen} betrachtet werden.

\begin{tikzpicture}[declare function={H(\x)=-1*(\x*log2{\x} + (1-\x)*log2{(1-\x)});}]
\begin{axis}%
[
    grid=major,     
    xmin=0,
    xmax=1,
    axis x line=bottom,
    ytick={0,.5,1},
    ymax=1,
    axis y line=middle,
    samples=100,
    domain=0:1,
    legend style={at={(1,0.9)}},
    x label style={at={(axis description cs:0.5,-0.1)},anchor=north},
    y label style={at={(axis description cs:-0.1,.5)},rotate=90,anchor=south},
    ylabel={$H(X)$},
    xlabel={$P(X=1)$}
]
    \addplot[blue,mark=none]   (x,{H(x)});
    \legend{$H(X)$}
\end{axis}
\end{tikzpicture}

Besteht $X$ nur aus einer Klasse so ist die Entropy 0. Sind beide Klassen gleich oft in $X$ vertreten, ist die Entropy maximal ($H(X) = 1$). Umso grösser die Entropy ist, desto grösser die \textbf{Unordnung} bezüglich der Klassen im Datenset.\\


Nun da wir wissen wie die Entropy berechnet wird, können wir uns der \textbf{Bestimmung der Aufteilung} widmen. Wir gehen wie folgt vor:

\begin{itemize}
	\item Berechnen der Entropy \textbf{vor und nach einem potentiellen Split}.
	\item Der Split, welcher die \textbf{Unordnung (Entropy) am meisten reduziert} wird priorisiert.
\end{itemize}


Bei unserem Beispiel haben wir drei Möglichkeiten für den ersten Split:
\begin{itemize}
	\item Ist die Person klein?
	\item Ist die Person leicht?
	\item Arbeitet die Person in einem Tech-Umfeld?
\end{itemize}
Dafür rechnen wir für jeden möglichen Split den \textbf{Information Gain} aus.
$$ \textbf{Information Gain} = \text{Entropy des Parent Node} - \text{Durchschnitt der Entropy der child-nodes} $$

\newpage

Die Entropy der Parent Nodes ist jeweils 1 wie wir bereits zuvor gezeigt haben. \\

\textbf{Ist die Person (nicht) klein?}
\begin{figure}[H]
	\centering
	\begin{forest}
	for tree={
		draw, 
		s sep=3em,
		parent anchor=south,
        child anchor=north,
        align=center,
        inner sep=1pt,
	}
	[klein $\leq 0.5$ \\ \text{Entropy = 1} \\ \text{\textbar X\textbar = 8 (Frau:4, Mann:4)}
	    [\text{Entropy = 0.811} \\ \text{\textbar X\textbar = 4 (Frau:1, Mann:3)}]
	    [\text{Entropy = 0.811} \\ \text{\textbar X\textbar = 4 (Frau:3, Mann:1)}]
	]
	\end{forest}
\end{figure}

Entropy der Child-Nodes ist jeweils:
$$ H(X) = -\left(\frac{1}{4} * log_{2}\left(\frac{1}{4}\right) + \frac{3}{4} * log_{2}\left(\frac{3}{4}\right)\right) = 0.811 $$

Der Information Gain:

$$ 1 - \frac{1}{2}*(0.811 + 0.811) = 0.189 $$

\textbf{Ist die Person (nicht) leicht?}
\begin{figure}[H]
	\centering
	\begin{forest}
	for tree={
		draw, 
		s sep=3em,
		parent anchor=south,
        child anchor=north,
        align=center,
        inner sep=1pt,
	}
	[klein $\leq 0.5$ \\ \text{Entropy = 1} \\ \text{\textbar X\textbar = 8 (Frau:4, Mann:4)}
	    [\text{Entropy = 0.811} \\ \text{\textbar X\textbar = 4 (Frau:1, Mann:3)}]
	    [\text{Entropy = 0.811} \\ \text{\textbar X\textbar = 4 (Frau:3, Mann:1)}]
	]
	\end{forest}
\end{figure}

Entropy der Child-Nodes ist jeweils:
$$ H(X) = -\left(\frac{1}{4} * log_{2}\left(\frac{1}{4}\right) + \frac{3}{4} * log_{2}\left(\frac{3}{4}\right)\right) = 0.811 $$

Der Information Gain:

$$ 1 - \frac{1}{2}*(0.811 + 0.811) = 0.189 $$


\newpage
\textbf{Arbeitet die Person (nicht) in Tech?}
\begin{figure}[H]
	\centering
	\begin{forest}
	for tree={
		draw, 
		s sep=3em,
		parent anchor=south,
        child anchor=north,
        align=center,
        inner sep=1pt,
	}
	[arbeitet in Tech $\leq 0.5$ \\ \text{Entropy = 1} \\ \text{\textbar X\textbar = 8 (Frau:4, Mann:4)}
	    [\text{Entropy = 0.722} \\ \text{\textbar X\textbar = 5 (Frau:4, Mann:1)}]
	    [\text{Entropy = 0} \\ \text{\textbar X\textbar = 3 (Frau:0, Mann:3)}]
	]
	\end{forest}
\end{figure}

Entropy des linken Child-Nodes ist:
$$ H(X) = -\left(\frac{4}{5} * log_{2}\left(\frac{4}{5}\right) + \frac{1}{5} * log_{2}\left(\frac{1}{5}\right)\right) = 722 $$

Entropy des rechten Child-Nodes ist:
$$ H(X) = -\left(0 + \frac{3}{3} * log_{2}\left(\frac{3}{3}\right)\right) = 0 $$

Der Information Gain:

$$ 1 - \frac{1}{2}*(0 + 0.722) = 0.639 $$

Den Split auf diesem Feature gibt uns also den höchsten Information Gain (reduziert die Unordnung im Datenset am Meisten). Daher wurde dieses Feature vom DT in Abbildung \ref{fig:dt_split} gewählt. \\

Diese Verfahren würde nun für die verbleibenden Samples im linken Child-Node weitergeführt werden.

% =========
\newpage
\subsubsection{Aufteilungsarten}

Die Aufteilung einer Decision kann in zwei Arten erfolgen. 

\textbf{multi-way split:}\\
\begin{figure}[H]
	\centering
	\label{fig:dt_multi}
	\begin{forest}
	for tree={draw, s sep=3em}
	[Car Type
	    [Family]
	    [Sports]
	    [Luxury]
	]
	\end{forest}
	\caption{Decision Tree multi-way Splitt}
\end{figure}


\textbf{2-way split (binär):}\\
Diese Variante wird meistens verwendet und ist einfacher zu optimieren.

\begin{figure}[H]
	\centering
	\label{fig:dt_binary}
	\begin{forest}
	for tree={draw, s sep=3em}
	[Car Type Family?
	    [Family]
	    [Car Type
	    	[Sports]
	    	[Luxury]
	    ]
	]
	\end{forest}
	\caption{Decision Tree 2-way Splitt}
\end{figure}


\subsubsection{Stop Kriterium}

Im idealen Fall sind nach einigen Entscheidungen nur noch Samples derselben Klasse in einem Branch übrig.  Jedoch ist dies in der Praxis selten der Fall. \\

Aus diesem Grund müssen wir definieren, wann der DT mit der Aufteilung (erstellen neuer internal nodes) aufhören soll. Dafür dient der \textbf{Hunt's} Algorithmus. \\

Dieser nutzt die folgenden Stop-Kriterien beim Erstellen des Baumes.

\begin{itemize}
	\item Alle Samples in einem Knoten gehören derselben Klasse an.
	\item Alle Samples in einem Knoten haben ähnliche Attribute (Features).
	\item Eine minimale (zuvor spezifizierte) Anzahl an Samples sind in einem Knoten vorhanden.
\end{itemize}

Die letzte Bedingung sorgt dafür, dass wir nicht einen riesigen Baum erstellen, welcher schlussendlich je nur ein Sample pro externen Knoten hat.

\newpage
\subsection{Vor und Nachteile des Decision Trees}

\subsubsection{Vorteile}

\begin{itemize}
	\item \textbf{Interpretierbar}
	\begin{itemize}
		\item Kann mathematisch zu 100 Prozent nachvollzogen werden warum wann welche Decision getroffen wird.
		\item Neue (ungesehene) Samples könnten von Hand durch die Entscheidungen des Trees durchlaufen werden. (quasi \textbf{manuelles debugging}).
	\end{itemize}
	\item \textbf{Skalierbar}
	\begin{itemize}
		\item Durch Verwendung des Hunt's Algo
		\item Funktioniert auch für eine sehr grosse Anzahl Features gut.
		\item Gute Geschwindigkeit des Lernens und Verarbeiten von Samples
	\end{itemize}
	\item \textbf{Einfach zu implementieren}
\end{itemize}


\subsubsection{Nachteile}

Der grösste und schwerwiegendste Nachteil des Decision Trees ist die Tendenz zum \textbf{Overfitting}. \\

Umso grösser der Tree wird (mehr Knoten) desto schlechter wird die Genauigkeit auf neuen Daten. Abbildung \ref{fig:dt_overfitting} zeigt die zunehmende Kluft die sich zwischen Training- und Testdaten ergibt.

\begin{figure}[h!]
	\includegraphics[scale=0.6]{figures/decision_tree_overfitting}
	\caption{Decision Tree Overfitting}
	\label{fig:dt_overfitting}
\end{figure}

\newpage
\subsubsection{Overfitting verhindern}

Um die Kluft zwischen Training- und Testdaten bezüglich der Genauigkeit zu verhindern, können diese Strategien angewendet werden: \\

\textbf{Limitieren der Tiefe (Anzahl Layer im Tree)}\\
Dies zwingt den DT Attribute zu verwenden, die über das ganze Datenset hinweg charakteristische Relevanz haben. Und \textbf{nicht} Attribute verwenden, die nur für sehr wenige Samples zutreffen.



\textbf{Eine minimale Anzahl Samples per externen Knoten fordern}\\
Verfolgt das gleiche Ziel wie oben, jedoch ohne explizit die Tiefe zu Limitieren. Da in jedem externen Knoten eine Mindestanzahl gefordert wird, kann der DT keine all zu spezifischen Decisions definieren.



% =========
\subsection{Python Model}

Hier ein Beispiel eines Decision Tree Models erstellt in Python mittels Sklearn.

\begin{lstlisting}[language=Python]
#!/usr/bin/env python
from sklearn import tree


# Data preparation
model = tree.DecisionTreeClassifier(
	criterion='entropy', # The function to measure the quality of a split. ('gini' or 'entropy')
	splitter='best', # The strategy used to choose the split at each node. ('best' or 'random')
	max_depth=None, # The maximum depth of the tree.
	min_samples_split=2, # The minimum number of samples required to split an internal node
	min_samples_leaf=1, # The minimum number of samples required to be at a leaf node.
	max_features=None, # The number of features to consider when looking for the best split
	max_leaf_nodes=None, # Max number of leaf nodes
	)
model.fit(X,y)
# ...
model.predict(sample)

\end{lstlisting}






\newpage
\section{Random Forest}

Der Random Forest ist ein \textbf{supervised learning Algorithmus}. Er kann sowohl für Klassifizierung wie auch für Regression verwendet werden. Wir konzentrieren uns hierbei aber lediglich auf die \textbf{Klassifizierung}. \\

Random Forest nutz die sogenannte \textbf{Ensemble Learning Methode}. Beim Ensemble Learning werden mehrere Lernalgorithmen zusammen verwendet, um ein \textbf{besseres Ergebnis} zu erhalten. Random Forest nutzt mehrere \textbf{Decision Trees}, um die \textbf{Tendenz zum Overfitting zu verringern}. \\

\textbf{Grundlegende Idee:}\\
\begin{itemize}
	\item Mehrere Decision Trees werden \textbf{zufällig} erstellt.
	\item Jeder DT mache eine Vorhersage.
	\item Die Vorhersage des RF ist eine \textbf{Kombination} der einzelnen Trees.
\end{itemize}

%
\begin{figure}[H]
	\centering
	\label{fig:random_forest}
	\begin{forest}
	for tree={
		l sep=2em, 
		s sep=2em, 
		anchor=center, 
		inner sep=0.3em, 
		fill=blue!50,
		circle, 
		%font=\Large\sffamily,
		where level=1{no edge}{},
		}
	[Training Data, draw, rectangle, rounded corners, orange, text=white,alias=TD
	    [,red!70,alias=a1
	    	[
	    		[,alias=a2]
	    		[]
	    	]
	    	[,red!70,edge label={node[above=1ex,marrow]{}}
	    		[
	    			[]
	    			[]
	    		]
	    		[,red!70,edge label={node[above=1ex,marrow]{}}
	    			[,red!70,edge label={node[below=1ex,marrow]{}}]
	    			[,alias=a3]
	    		]
	    	]
	    ]
	    [,red!70,alias=b1
	    	[,red!70,edge label={node[below=1ex,marrow]{}}
	    		[
	    			[,alias=b2]
	    			[]
	    		]
	    		[,red!70,edge label={node[above=1ex,marrow]{}}]
	    	]
	    	[
	    		[]
	    		[
	    			[]
	    			[,alias=b3]
	    		]
	    	]
	    ]
	    [~$\cdots$~,scale=3,no edge,fill=none,yshift=-1em]
	    [,red!70,alias=c1
	    	[
	    		[,alias=c2]
	    		[]
	    	]
	    	[,red!70,edge label={node[above=1ex,marrow]{}}
	    		[,red!70,edge label={node[above=1ex,marrow]{}}
	    			[,alias=c3]
	    			[,red!70,edge label={node[above=1ex,marrow]{}}]
	    		]
	    		[,alias=c4]
	    	]
	    ]
	]
	\node[draw,fit=(a1)(a2)(a3)](f1){};  
	\node[draw,fit=(b1)(b2)(b3)](f2){};  
	\node[draw,fit=(c1)(c2)(c3)(c4)](f3){};
	\node[below right=0.5em, inner sep=0pt] at (f1.north west) {Tree 1};
	\node[below right=0.5em, inner sep=0pt] at (f2.north west) {Tree 2};
	\node[below right=0.5em, inner sep=0pt] at (f3.north west) {Tree $n$};
	\path (f1.south west)--(f3.south east) node[midway,below=4em, node box] (myResultNode) {Mittelwert bei Regression. Mehrheitsbeschluss bei Klassifikation};
	\node[below=2em of myResultNode, node box] (pred){Prediction};
	\foreach \X in {1,2,3}{
		\draw[-stealth] (TD) -- (f\X.north);
		\draw[-stealth] (f\X.south) -- (myResultNode);
	};
	\draw[-stealth] (myResultNode) -- (pred);
	\end{forest}
	\caption{Random Forest}
\end{figure}
%

% ==================
\newpage
\subsection{Zufälligkeit im Random Forest}

Es gibt \textbf{zwei Hauptarten}, um den Random Forest zu erstellen. \textbf{Bagging} und \textbf{Random Vector Method}. Diese können zusammen oder auch einzeln beim Erstellen des RF-Models verwendet werden. 

\subsubsection{Bagging}

Diese Methode ist auch als \textbf{Bootstraping} bekannt.

\begin{itemize}
	\item Jeder DT nutzt nur eine \textbf{Teilmenge (bootstrap sample)} der Trainingsdaten.
	\item Somit können sich die einzelnen DTs nicht komplett den Trainisdaten anpassen (weniger overfitting). 
\end{itemize}

%
\begin{figure}[H]
	\centering
	\label{fig:random_forest_bagging}
	\begin{forest}
	for tree={
		l sep=2em, 
		s sep=2em, 
		anchor=center, 
		inner sep=0.3em, 
		fill=blue!50,
		circle, 
		%font=\Large\sffamily,
		where level=1{no edge}{},
		}
	[Training Data, draw, rectangle, rounded corners, orange, text=white,alias=TD
	    [Bootstrap Set 1, rectangle, text=white, alias=BS1
	    [,red!70,alias=a1
	    	[
	    		[,alias=a2]
	    		[]
	    	]
	    	[,red!70,edge label={node[above=1ex,marrow]{}}
	    		[
	    			[]
	    			[]
	    		]
	    		[,red!70,edge label={node[above=1ex,marrow]{}}
	    			[,red!70,edge label={node[below=1ex,marrow]{}}]
	    			[,alias=a3]
	    		]
	    	]
	    ]]
	    [Bootstrap Set 2, rectangle, text=white, alias=BS2
	    [,red!70,alias=b1
	    	[,red!70,edge label={node[below=1ex,marrow]{}}
	    		[
	    			[,alias=b2]
	    			[]
	    		]
	    		[,red!70,edge label={node[above=1ex,marrow]{}}]
	    	]
	    	[
	    		[]
	    		[
	    			[]
	    			[,alias=b3]
	    		]
	    	]
	    ]]
	    [~$\cdots$~,scale=3,no edge,fill=none,yshift=-1em]
	    [Bootstrap Set n, rectangle, text=white, alias=BSn
	    [,red!70,alias=c1
	    	[
	    		[,alias=c2]
	    		[]
	    	]
	    	[,red!70,edge label={node[above=1ex,marrow]{}}
	    		[,red!70,edge label={node[above=1ex,marrow]{}}
	    			[,alias=c3]
	    			[,red!70,edge label={node[above=1ex,marrow]{}}]
	    		]
	    		[,alias=c4]
	    	]
	    ]]
	]
	\node[draw,fit=(a1)(a2)(a3)](f1){};  
	\node[draw,fit=(b1)(b2)(b3)](f2){};  
	\node[draw,fit=(c1)(c2)(c3)(c4)](f3){};
	\node[below right=0.5em, inner sep=0pt] at (f1.north west) {Tree 1};
	\node[below right=0.5em, inner sep=0pt] at (f2.north west) {Tree 2};
	\node[below right=0.5em, inner sep=0pt] at (f3.north west) {Tree $n$};
	\path (f1.south west)--(f3.south east) node[midway,below=4em, node box] (myResultNode) {Mittelwert bei Regression. Mehrheitsbeschluss bei Klassifikation};
	\node[below=2em of myResultNode, node box] (pred){Prediction};
	\draw[-stealth] (TD) -- (BS1.north);
	\draw[-stealth] (TD) -- (BS2.north);
	\draw[-stealth] (TD) -- (BSn.north);
	\foreach \X in {1,2,3}{
		%\draw[-stealth] (TD) -- (f\X.north);
		\draw[-stealth] (f\X.south) -- (myResultNode);
	};
	\draw[-stealth] (myResultNode) -- (pred);
	\end{forest}
	\caption{Random Forest: Bagging}
\end{figure}
%

\subsubsection{Random Vector Method}

\begin{itemize}
	\item Bei \textbf{jeder Entscheidung} (internal Node) wird der beste Split aus $m$ \textbf{zufälligen Attributen} (Features) gewählt. Anstatt von allen möglichen Features das Beste zu wählen.
	\item Die führt bei jedem Baum zu anderen Entscheidungsknoten.
\end{itemize}




\newpage
\subsection{Algorithmus}
Hier eine vereinfachte Darstellung des Random Forest Algorithmus.\\

\begin{algorithm}
	\caption{Random Forest Algorithmus}
	\begin{algorithmic} 
	\For{$b \gets 1$ to $B$}
	\Comment B = Anzahl Trees
	\State {Ziehe eine Bootstrap-Stichprobe aus den Trainingsdaten}
	\State {Erstelle einen Decision Tree $T_{b}$}
	\Repeat
	\Comment Rekursiv für jeden Endknoten
		\State {Wähle zufällig $m$ Variablen (Features) aus}
		\State {Wähle den besten Split-Punkt/die beste Variable unter den $m$ Variablen aus.}
		\State {Teile den Knoten in zwei Child-nodes}
	\Until{Gewünschte Grösse des Baumes erreicht ist}
	\EndFor
	\Return Gebe das Tree-Ensemble zurück
	\end{algorithmic}
\end{algorithm}


% ===
\subsection{Python Model}

Hier ein Beispiel wie ein Random Forest Model in Python mittels Sklearn erstellt werden könnte. \\

\begin{lstlisting}[language=Python]
#!/usr/bin/env python


from sklearn.ensemble import RandomForestClassifier

# Data preparation

model = RandomForestClassifier(
    n_estimators=100, # The number of trees in the forest.
    criterion='entropy', # The function to measure the quality of a split. 
    max_depth=None, # The maximum depth of the tree.
    max_features='sqrt', # The minimum number of samples required to be at a leaf node
    bootstrap=True, # Whether bootstrap samples are used when building trees
    max_samples=None # If bootstrap is True, the number of samples to draw from X to train each base estimator.
)

model.fit(X,y)
# ...
model.predict(sample)

\end{lstlisting}






\newpage
\section{Neuronale Netzwerke}
\subsection{Representation}
\begin{flushleft}


Der Inputnode $x_{0}$ wird nicht immer eingezeichnet und repräsentiert den Bias-Node.

Weiter gilt:

$\begin{aligned}
 \alpha_{i}^{(j)} &= \text{Aktivierung der Einheit i im Layer j}
\end{aligned}$

$\begin{aligned}
 \Theta^{(j)} &= \text{Gewichtsmatrix welche Layer j auf Layer j+1 mapt}
\end{aligned}$

\begin{tikzpicture}[
     % define styles 
     clear/.style={ 
         draw=none,
         fill=none
     },
     net/.style={
         matrix of nodes,
         nodes={ draw, circle, inner sep=10pt },
         nodes in empty cells,
         column sep=2cm,
         row sep=-9pt
     },
     >=latex
]
% define matrix mat to hold nodes
% using net as default style for cells
\matrix[net] (mat)
{
% Define layer headings
|[clear]| \parbox{1.3cm}{\centering Input\\layer} 
    & |[clear]| \parbox{1.3cm}{\centering Hidden\\layer} 
    & |[clear]| \parbox{1.3cm}{\centering Output\\layer} \\
         
$x_{0}$  & |[clear]|        & |[clear]| \\
|[clear]|         & $\alpha_{1}^{(2)}$ & |[clear]| \\
$x_{1}$  & |[clear]|        & |[clear]| \\
|[clear]|         & |[clear]|        & |[clear]| \phantom{$a_{0}^{0}$} \\
$x_{2}$  & $\alpha_{2}^{(2)}$ & $$ \\
|[clear]|         & |[clear]|        & |[clear]|  \phantom{$a_{0}^{0}$} \\
$x_{3}$  & |[clear]|        & |[clear]| \\
|[clear]|         & $\alpha_{3}^{(2)}$ & |[clear]| \\
$x_{4}$  & |[clear]|        & |[clear]| \\ 
};
% left most lines into input layers
\foreach \ai in {2,4,6,8,10}
    \draw[<-] (mat-\ai-1) -- +(-2cm,0);
% lines from a_{i}^{0} to each a_{j}^{1}
\foreach \ai in {2,4,6,8,10} {
    \foreach \aii in {3,6,9}
        \draw[->] (mat-\ai-1) -- (mat-\aii-2);
        }
% lines from a_{i}^{1} to a_{0}^{2}
\foreach \ai in {3,6,9}
  \draw[->] (mat-\ai-2) -- (mat-6-3);
    
% right most line with Output label
\draw[->] (mat-6-3) -- node[above] {$h_{\Theta}(x)$} +(2cm,0);
\end{tikzpicture}




$$ \alpha_{1}^{(2)} = g(\Theta_{10}^{(1)}x_{0} +  \Theta_{11}^{(1)}x_{1} + \Theta_{12}^{(1)}x_{2} +\Theta_{13}^{(1)}x_{3} + \Theta_{14}^{(1)}x_{4})$$

$$ \alpha_{2}^{(2)} = g(\Theta_{20}^{(1)}x_{0} +  \Theta_{21}^{(1)}x_{1} + \Theta_{22}^{(1)}x_{2} +\Theta_{23}^{(1)}x_{3} + \Theta_{24}^{(1)}x_{4})$$

$$ \alpha_{3}^{(2)} = g(\Theta_{30}^{(1)}x_{0} +  \Theta_{31}^{(1)}x_{1} + \Theta_{32}^{(1)}x_{2} +\Theta_{33}^{(1)}x_{3} + \Theta_{34}^{(1)}x_{4})$$


$$ h_{\Theta}(x) = \alpha_{1}^{(3)} = g(\Theta_{10}^{(2)}\alpha_{0}^{(2)} + \Theta_{11}^{(2)}\alpha_{1}^{(2)} + \Theta_{12}^{(2)}\alpha_{2}^{(2)} + \Theta_{13}^{(2)}\alpha_{3}^{(2)}) $$


Existiert ein neuronales Netz mit $s_{j}$ Einheiten im Layer $j$, $s_{j+1}$ Einheiten im Layer $j+1$, dann hat $\Theta^{(j)}$ die Dimension $s_{j+1} \times (s_{j} + 1)$.

\end{flushleft}


\subsection{Logik Beispiel}

\subsubsection{AND}
\begin{flushleft}


Ein AND-Gatter kann wie folgt erstellt werden:

\begin{tikzpicture}[
     % define styles 
     clear/.style={ 
         draw=none,
         fill=none
     },
     net/.style={
         matrix of nodes,
         nodes={ draw, circle, inner sep=10pt },
         nodes in empty cells,
         column sep=2cm,
         row sep=-9pt
     },
     >=latex
]
% define matrix mat to hold nodes
% using net as default style for cells
\matrix[net] (mat)
{
% Define layer headings
|[clear]| \parbox{1.3cm}{\centering Input\\layer} & |[clear]| \parbox{1.3cm}{\centering Output\\layer} \\
         
$+1$  		& |[clear]| \\
|[clear]| 	& |[clear]| \\
$x_{1}$  	& |[clear]| \\
|[clear]| 	& $$ \\
$x_{2}$  	& |[clear]| \\
};
\draw[->] (mat-2-1) -- node[above=1mm] {-30} (mat-5-2);
\draw[->] (mat-4-1) -- node[above=1mm] {20} (mat-5-2);
\draw[->] (mat-6-1) -- node[above=1mm] {20} (mat-5-2);
\draw[->] (mat-5-2) -- node[above] {$h_{\Theta}(x)$} +(2cm,0);
\end{tikzpicture}

$$ h_{\Theta}(x) = g(-30 + 20x_{1} + 20x_{2}) $$


\begin{center}
\begin{tabular}{ c c|r } 

 $x_{1}$ & $x_{2}$ & $h_{\Theta}(x)$ \\ 
  \hline
 0 & 0 & $g(-30) \approx 0$ \\ 
 0 & 1 & $g(-10) \approx 0$ \\ 
 1 & 0 & $g(-10) \approx 0$ \\ 
 1 & 1 & $g(10) \approx 1$ \\ 

\end{tabular}
\end{center}
\end{flushleft}


\subsubsection{OR}
\begin{flushleft}


Ein OR-Gatter kann wie folgt erstellt werden:

\begin{tikzpicture}[
     % define styles 
     clear/.style={ 
         draw=none,
         fill=none
     },
     net/.style={
         matrix of nodes,
         nodes={ draw, circle, inner sep=10pt },
         nodes in empty cells,
         column sep=2cm,
         row sep=-9pt
     },
     >=latex
]
% define matrix mat to hold nodes
% using net as default style for cells
\matrix[net] (mat)
{
% Define layer headings
|[clear]| \parbox{1.3cm}{\centering Input\\layer} & |[clear]| \parbox{1.3cm}{\centering Output\\layer} \\
         
$+1$  		& |[clear]| \\
|[clear]| 	& |[clear]| \\
$x_{1}$  	& |[clear]| \\
|[clear]| 	& $$ \\
$x_{2}$  	& |[clear]| \\
};
\draw[->] (mat-2-1) -- node[above=1mm] {-10} (mat-5-2);
\draw[->] (mat-4-1) -- node[above=1mm] {20} (mat-5-2);
\draw[->] (mat-6-1) -- node[above=1mm] {20} (mat-5-2);
\draw[->] (mat-5-2) -- node[above] {$h_{\Theta}(x)$} +(2cm,0);
\end{tikzpicture}

$$ h_{\Theta}(x) = g(-10 + 20x_{1} + 20x_{2}) $$


\begin{center}
\begin{tabular}{ c c|r } 

 $x_{1}$ & $x_{2}$ & $h_{\Theta}(x)$ \\ 
  \hline
 0 & 0 & $g(-10) \approx 0$ \\ 
 0 & 1 & $g(10) \approx 1$ \\ 
 1 & 0 & $g(10) \approx 1$ \\ 
 1 & 1 & $g(30) \approx 1$ \\ 

\end{tabular}
\end{center}
\end{flushleft}


\subsubsection{NOT}
\begin{flushleft}


Ein NOT-Gatter kann wie folgt erstellt werden:

\begin{tikzpicture}[
     % define styles 
     clear/.style={ 
         draw=none,
         fill=none
     },
     net/.style={
         matrix of nodes,
         nodes={ draw, circle, inner sep=10pt },
         nodes in empty cells,
         column sep=2cm,
         row sep=-9pt
     },
     >=latex
]
% define matrix mat to hold nodes
% using net as default style for cells
\matrix[net] (mat)
{
% Define layer headings
|[clear]| \parbox{1.3cm}{\centering Input\\layer} & |[clear]| \parbox{1.3cm}{\centering Output\\layer} \\
         
$+1$  		& |[clear]| \\
|[clear]| 	& $$ \\
$x_{1}$  	& |[clear]| \\
};
\draw[->] (mat-2-1) -- node[above=1mm] {10} (mat-3-2);
\draw[->] (mat-4-1) -- node[above=1mm] {-20} (mat-3-2);
\draw[->] (mat-3-2) -- node[above] {$h_{\Theta}(x)$} +(2cm,0);
\end{tikzpicture}

$$ h_{\Theta}(x) = g(10 - 20x_{1}) $$


\begin{center}
\begin{tabular}{ c|r } 

 $x_{1}$ & $h_{\Theta}(x)$ \\ 
  \hline
 0 & $g(10) \approx 1$ \\ 
 1 & $g(-10) \approx 0$ \\ 

\end{tabular}
\end{center}
\end{flushleft}



\subsubsection{XNOR}
\begin{flushleft}


Durch Kombinationen von einzelnen NN können komplexere Gatter erstellt werden. Wie bspw. das XNOR Gatter.

\begin{tikzpicture}[
     % define styles 
     clear/.style={ 
         draw=none,
         fill=none
     },
     net/.style={
         matrix of nodes,
         nodes={ draw, circle, inner sep=10pt },
         nodes in empty cells,
         column sep=2cm,
         row sep=-9pt
     },
     >=latex
]
% define matrix mat to hold nodes
% using net as default style for cells
\matrix[net] (mat)
{
% Define layer headings
|[clear]| \parbox{1.3cm}{\centering Input\\layer} 
	& |[clear]| \parbox{1.3cm}{\centering Hidden\\layer} 
	& |[clear]| \parbox{1.3cm}{\centering Output\\layer} \\
         
$+1$  		& $+1$ 					&	|[clear]|\\
|[clear]|	& |[clear]|				&	|[clear]| \\
$x_{1}$  	& $\alpha_{1}^{(2)}$  	&  	$\alpha_{1}^{(3)}$\\
|[clear]|	& |[clear]|				&	|[clear]| \\
$x_{2}$  	& $\alpha_{1}^{(2)}$ 	&  	|[clear]|\\
};
% +1
\draw[->] (mat-2-1) -- node[above=0mm] {-30} (mat-4-2);
\draw[->] (mat-2-1) -- node[above=0mm] {10} (mat-6-2);
%  x1
\draw[->] (mat-4-1) -- node[below=0mm] {20} (mat-4-2);
\draw[->] (mat-4-1) -- node[above=0mm] {-20} (mat-6-2);
%  x2
\draw[->] (mat-6-1) -- node[below=0mm] {20} (mat-4-2);
\draw[->] (mat-6-1) -- node[below=0mm] {-10} (mat-6-2);

%
\draw[->] (mat-2-2) -- node[above=0mm] {-10} (mat-4-3);
\draw[->] (mat-4-2) -- node[above=0mm] {20} (mat-4-3);
\draw[->] (mat-6-2) -- node[above=0mm] {20} (mat-4-3);

\draw[->] (mat-4-3) -- node[above] {$h_{\Theta}(x)$} +(2cm,0);

\end{tikzpicture}


\begin{center}
\begin{tabular}{ c c|c c|r } 

 $x_{1}$ & $x_{2}$ & $\alpha_{1}^{(2)}$ & $\alpha_{2}^{(2)}$ & $h_{\Theta}(x)$ \\ 
  \hline
 0 & 0 & 0 & 1 & 1 \\ 
 0 & 1 & 0 & 0 & 0 \\ 
 1 & 0 & 0 & 0 & 0 \\ 
 1 & 1 & 1 & 0 & 1 \\ 

\end{tabular}
\end{center}
\end{flushleft}





% ============================================================



% ============================================================


%\newpage
\section{Tryout}\label{sec:tryout}

\begin{figure}[H]	% H stands for here (place right here)
	\centering
	\includegraphics[height=5cm]{figures/tryout.png}
	\caption[Optional optional]{Entscheidungsbaum}
	\label{fig:tryout}
\end{figure}

Wie auf der Abbildung \ref{fig:tryout} zu sehen ist.....


\begin{table}[H]
	\centering
	\label{tab:tryouttab}
\caption[This is an optional caption, without reference]{Local caption, with reference}
	\cite{ref:ds_1, ref:nn_1, ref:ai_1}	% Used to add cites (zitieren)

	\begin{tabular}{l c r}
		Area & Number of rooms & Price \\ \hline
		80	& 4				& 1680 \\
		100	& 5				& 2300 \\
		50	& 2.5				& 1500 \\

	\end{tabular}
\end{table}


\begin{itemize}
	\item This is an item
	\item This is another item
	\begin{itemize}
		\item This is a further item
		\item [blub] This is an item with a custom bullet point
	\end{itemize}
\end{itemize}

\begin{enumerate}
	\item This is a numbered item
	\item And so on
\end{enumerate}


\newpage
\subsection{Math examples}

Here's an example within a sentence $E =mc^2$.

And here one example $$a=v/t$$ which is centred. \\

$$-\frac{\hbar^2}{2m}\frac{d^2\Psi}{dx^2} = E\Psi$$

Fractions

$$d = v_it + \frac{1}{2} \cdot at^2$$
$$d = v_it + \sfrac{1}{2} \cdot at^2$$


Brackets:
$$\left( \frac{1}{2} \right) \cdot 2 = 1$$	% use \left( ..... \right) to match the brackets to the content
$$\left| -7 \right| = 7$$
$$\sqrt{4} = 2$$
$$\sqrt{4} \ne 1$$
$$\sqrt{4} < 5$$
$$ \pi \approx 3 $$
$$ \pi \times \sqrt{4} < 15 $$

\begin{eqnarray}	% Equation array
	3x + 14 &=& 20 \\
	3x &=& 6 \\
	x &=& 2
\end{eqnarray}

\begin{equation}
\label{eq:first}
x^2 + 3x - 7 = 0
\end{equation}

\newpage
\subsection{Graphs}




\begin{tikzpicture}[sibling distance=12em,
							%root/.style={treenode,circle,draw},
							every child node/.style={circle, draw=black},
							]
%	[align=center, sibling distance=5cm]
	\node[fill=black]{}
		child { node {B}
		[sibling distance=6em]
			child { node{A $\cap$ B}  edge from parent node[left] {$P(A|B)$} }
			child { node{$\bar{A} \cap B$} edge from parent node[right] {$P(\bar{A}|B)$}}
			}
		child{ node{$\bar{B}$}
			[sibling distance=6em]
			child{ node{$A \cap \bar{B}$} edge from parent node[left] {$P(A|\bar{B})$}}
			child{ node{$\bar{A} \cap \bar{B}$} edge from parent node[right] {$P(\bar{A}|\bar{B})$} }
		       }
	;

\end{tikzpicture}
% ****************************************************************

% ============================================================




% REFERENCES ============================================================
\cleardoublepage
%\renewcommand{\bibname}{Referenzen}	% Rename the bibliography title
\bibliographystyle{IEEEtran}	% Adds cites (Zitate)
\addcontentsline{toc}{section}{Referenzen}
% references can easily be generated using the OS X tool "BibDesk"
% Make sure you build your LaTeX document in BibTeX after defining your references
%	in order to make them valid.
\bibliography{references/book_ref1}
% ============================================================

% APPENDIX ============================================================
\cleardoublepage
\appendix
\section{Cheatsheet}
% ============================================================


\end{document}