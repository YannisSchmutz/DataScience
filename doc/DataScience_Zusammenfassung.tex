\documentclass{article}

% PACKAGES  ============================================================
\usepackage{lipsum}		% Used to generate dummy-text (\lipsum[1])
\usepackage[margin=2.54cm,includefoot]{geometry}		% Used to control margins
\usepackage{scrextend}	% To be able to use \begin{addmargin}

\usepackage{graphicx} % allows to import images
\usepackage{float}	% allows for control of float positions
\usepackage{tikz} 	% Used to create trees
\usepackage{pgfplots}	% to create plots

\usepackage{forest} % Used to create trees

\usepackage{amsmath}	 % to use cases in equations

\usepackage[hidelinks]{hyperref}	% allows for clickable references

\usepackage[numbers,sort&compress]{natbib} 	% sorts the cites in increasing order automatically when referenced and compresses successive references

%\usepackage[applemac]{inputenc}
\usepackage[utf8]{inputenc}		% Be able to use umlaute

\usepackage[ngerman]{babel}
%\usepackage[english,german]{babel}		% Change language from english to german (order matters!)
%\usepackage{ngerman}

\usepackage{fancyhdr}	% Used for Header and Footer stuff

\usepackage{xfrac}	% allows for slanted fractions 
% ============================================================

% HEADER AND FOOTER STUFF ============================================================
\pagestyle{fancy}
\fancyhead{}	% clears header
\fancyfoot{}	% clears footer
\fancyfoot[R]{\thepage}	% sets position right
\renewcommand{\headrulewidth}{0pt}	% removes header line by setting it to zero
\renewcommand{\footrulewidth}{1pt}		% add footer line by setting it to one
% ============================================================

% DEFINE LIST ITEM BULLETS ============================================================
\renewcommand{\labelitemi}{$\bullet$}	% first list item
\renewcommand{\labelitemii}{$\circ$}	% one-indented item
\renewcommand{\labelitemiii}{$\diamond$}	% twice-indented item
% ============================================================


\begin{document}

% TITLE PAGE ============================================================
\begin{titlepage}
	
	\begin{center}
	\line(1,0){330} \\
	[2mm]
	\huge{\bfseries Data Science Zusammenfassung} \\
	[2mm]
	\line(1,0){320} \\
	[1,5cm]
	\textsc{\LARGE By Yannis Schmutz} \\
	[0.75cm]
	\textsc{\large todo} \\
	
	\end{center}
	
\end{titlepage}
% ============================================================

% PREFACE STUFF ============================================================
\pagenumbering{roman}		% sets the page numbering to roman for the preface etc.
\section*{Zusammenfassung}	% Adds a section without a number in front
\addcontentsline{toc}{section}{\numberline{}Zusammenfassung}	% adds a section without a number in front to the ToC
\cleardoublepage	% Finishes the current page so that the following page will always be odd.
% ****************************************************************

% TABLE OF CONTENTS ============================================================
\renewcommand{\contentsname}{Inhaltsverzeichnis}	% Rename table of contents to the german version
\tableofcontents		% adds table of contents (this needs to be compiled twice sometimes in order to update)
\thispagestyle{empty}	% removes header & footer on this page
\cleardoublepage	% Finishes the current page so that the following page will always be odd.
% ============================================================

% LIST OF FIGURES ============================================================
%\renewcommand{\listfigurename}{Abbildungsverzeichnis}	% renames list of figures
\listoffigures	% generates a list of figures
\addcontentsline{toc}{section}{Abbildungsverzeichnis}	% Adds list of figures to the ToC
\cleardoublepage
% ============================================================

% LIST OF TABLES ============================================================
%\renewcommand{\listtablename}{Tabellenverzeichnis}
\listoftables
\addcontentsline{toc}{section}{Tabellenverzeichnis}
\cleardoublepage
% ============================================================

% START OF REGULAR CHAPTERS ============================================================
\setcounter{page}{1}		% Sets this page to the first one (and not the table of contents)
\pagenumbering{arabic}	% Sets the page numbering back to arabic

%\include{chapters/E}
\newpage
\section{Statistik}
\label{sec:stat}
\lipsum[1]

% ****************************************************************
\newpage
\section{Probabilistik}
\subsection{Bedingte Wahrscheinlichkeit}

Die bedingte Wahrscheinlichkeit ist die Wahrscheinlichkeit des Eintreten eines Ereignisses A unter der Bedingung, dass die Wahrscheinlichkeit f�r das Eintreten eines Ereignisses B bereits bekannt ist. Man spricht von \dq{A} unter der Bedingung B\dq. Oder auch $P(A|B)$.\\


Sind zwei Ereignisse E, F voneinander \textbf{unabh�ngig}, so gilt:
$$ P(E \cap F) = P(E)P(F) $$
$$ P(E|F) = P(E)$$

Sind jedoch zwei Ereignisse A, B \textbf{nicht unabh�ngig} so lautet die Formel f�r A unter der Bedingung B:
	$$P(A|B) = \frac{P(A \cap B)}{P(B)}$$


Daraus erschliesst sich:
	$$P(A \cap B) = P(A|B)P(B)$$


Das Aufzeichnen eines Wahrscheinlichkeitsbaumes hilft zur Veranschaulichung: \\

\begin{figure}[H]
	\centering
	\label{fig:probability_tree}
	\begin{forest}
	%\label{fig:probability_tree}
	for tree={circle,draw, s sep=3em}
	[ 
	    [$B$,edge label={node[midway,left] {$P(B)$}}
	      [$A \cap B$,edge label={node[midway,left] {$P(A|B)$}} ] 
	      [$\bar{A} \cap B$,edge label={node[midway,right] {$P(\bar{A}|B)$}}] 
	    ]
	    [$\bar{B}$,edge label={node[midway,right] {$P(\bar{B})$}}
	      [$A \cap \bar{B}$, edge label={node[midway,left] {$P(A|\bar{B})$}}] 
	      [$\bar{A} \cap \bar{B}$, edge label={node[midway,right] {$P(\bar{A}| \bar{B})$}}] 
	  ] 
	]
	\end{forest}
	\caption{Wahrscheinlichkeitsbaum}
\end{figure}

\subsubsection{Satz von Bayes}
%Verweis auf Kapitel \pageref{sec:stat}.

Der Satz von Bayes zeigt den Zusammenhang zwischen $P(A|B)$ und $P(B|A)$ auf:

\begin{equation}
\label{eq:bayes_theorem}
	P(A|B) = \frac{P(A \cap B)}{P(B)} = \frac{P(B|A)P(A)}{P(B)}
\end{equation}

Diese Gleichung \ref{eq:bayes_theorem} entsteht, wenn man den Ausdruck $P(A \cap B)$ anhand den umgekehrten Wahrscheinlichkeitsbaums \ref{fig:inv_probability_tree} ausdr�ckt. 


\begin{figure}[H]
	\centering
	\label{fig:inv_probability_tree}
	\begin{forest}
	for tree={circle,draw, s sep=3em}
	[ 
	    [$A$,edge label={node[midway,left] {$P(A)$}}
	      [$A \cap B$,edge label={node[midway,left] {$P(B|A)$}} ] 
	      [$A \cap \bar{B}$,edge label={node[midway,right] {$P(\bar{B}|A)$}}] 
	    ]
	    [$\bar{A}$,edge label={node[midway,right] {$P(\bar{A})$}}
	      [$\bar{A} \cap B$, edge label={node[midway,left] {$P(B|\bar{A})$}}] 
	      [$\bar{A} \cap \bar{B}$, edge label={node[midway,right] {$P(\bar{B}| \bar{A})$}}] 
	  ] 
	]
	\end{forest}
	\caption{Umgekehrter Wahrscheinlichkeitsbaum}
\end{figure}
% ****************************************************************


% ============================================================
\newpage
\section{Machine Learning}
\label{sec:ml}

\subsection{Uebersicht}

ä

\begin{figure}[H]
	\centering
	\label{fig:ml_overview}
	\begin{forest}
	for tree={draw, s sep=3em}
	[Machine Learning
	    [Ueberwachtes Lernen
	        [Klassifizierung
	            [Naive Bayes]
	             [k-N-N]
	             [NN]
	        ]
	        [Regression
	            [Linear Regression]
	            [NN]
	        ]
	    ]
	    [Unueberwachtes Lernen
	        [Clustering
	            [K-means]
	        ]
	        [Assoziierung]
	    ]
	]
	\end{forest}
	\caption{Machine Learning Kategorien}
\end{figure}

% #################################
\subsubsection{Ueberwachtes Lernen}
\begin{flushleft}

Ueberwachte Lern-Algorithmen probieren Beziehungen und Abhaengigkeiten zwischen den Input-Features und des zu erzielenden Outputs zu erschliessen. Dies unter der Verwendung von \textbf{beschrifteten Daten}. Diese koennen zu Trainingszwecken verwendet werden. Jeder Satz an Daten besteht aus Input-Werten sowie einem dazugehoerigen bekannten Output-Wert. Nach dem Trainieren des Algorithmus versucht dieser anhand von \textbf{neuen} Input-Features den dazugehoerigen \textbf{unbekannten} Output vorherzusagen.
\linebreak

Das ueberwachte Lernen kann in zwei Kategorien unterteilt werden:

\begin{itemize}
	\item \textbf{Klassifizierung:} Ziel der Klassifizierung ist es, diskrete Werte vorherzusagen (bspw. Wahr/ Falsch, Spam-Mail/ normales Mail). 
	\item \textbf{Regression:} Das Ziel der Regression ist die Vorhersage kontinuierlicher Werte (bspw. Hauspreise in Abhaengigkeit von Flaeche und Anzahl Zimmer).
\end{itemize}

\end{flushleft}

\subsubsection{Unueberwachtes Lernen}
\begin{flushleft}

Im unueberwachten Lernen stehen den Algorithmen \textbf{keine} beschrifteten Daten zur Verfuegung. Die Algorithmen versuchen eigenstaendig Pattern in den zu behandelnden Daten zu erkennen und sie dadurch beispielsweise gruppieren zu koennen.
\end{flushleft}

% #################################
\subsection{Algorithmen}
\subsubsection{Naive Bayes}
\begin{flushleft}
\end{flushleft}

% äöüÄÖÜ

% ============================================================


%\newpage
\section{Tryout}\label{sec:tryout}

\begin{figure}[H]	% H stands for here (place right here)
	\centering
	\includegraphics[height=5cm]{figures/tryout.png}
	\caption[Optional optional]{Entscheidungsbaum}
	\label{fig:tryout}
\end{figure}

Wie auf der Abbildung \ref{fig:tryout} zu sehen ist.....


\begin{table}[H]
	\centering
	\label{tab:tryouttab}
\caption[This is an optional caption, without reference]{Local caption, with reference}
	\cite{ref:ds_1, ref:nn_1, ref:ai_1}	% Used to add cites (zitieren)

	\begin{tabular}{l c r}
		Area & Number of rooms & Price \\ \hline
		80	& 4				& 1680 \\
		100	& 5				& 2300 \\
		50	& 2.5				& 1500 \\

	\end{tabular}
\end{table}


\begin{itemize}
	\item This is an item
	\item This is another item
	\begin{itemize}
		\item This is a further item
		\item [blub] This is an item with a custom bullet point
	\end{itemize}
\end{itemize}

\begin{enumerate}
	\item This is a numbered item
	\item And so on
\end{enumerate}


\newpage
\subsection{Math examples}

Here's an example within a sentence $E =mc^2$.

And here one example $$a=v/t$$ which is centred. \\

$$-\frac{\hbar^2}{2m}\frac{d^2\Psi}{dx^2} = E\Psi$$

Fractions

$$d = v_it + \frac{1}{2} \cdot at^2$$
$$d = v_it + \sfrac{1}{2} \cdot at^2$$


Brackets:
$$\left( \frac{1}{2} \right) \cdot 2 = 1$$	% use \left( ..... \right) to match the brackets to the content
$$\left| -7 \right| = 7$$
$$\sqrt{4} = 2$$
$$\sqrt{4} \ne 1$$
$$\sqrt{4} < 5$$
$$ \pi \approx 3 $$
$$ \pi \times \sqrt{4} < 15 $$

\begin{eqnarray}	% Equation array
	3x + 14 &=& 20 \\
	3x &=& 6 \\
	x &=& 2
\end{eqnarray}

\begin{equation}
\label{eq:first}
x^2 + 3x - 7 = 0
\end{equation}

\newpage
\subsection{Graphs}




\begin{tikzpicture}[sibling distance=12em,
							%root/.style={treenode,circle,draw},
							every child node/.style={circle, draw=black},
							]
%	[align=center, sibling distance=5cm]
	\node[fill=black]{}
		child { node {B}
		[sibling distance=6em]
			child { node{A $\cap$ B}  edge from parent node[left] {$P(A|B)$} }
			child { node{$\bar{A} \cap B$} edge from parent node[right] {$P(\bar{A}|B)$}}
			}
		child{ node{$\bar{B}$}
			[sibling distance=6em]
			child{ node{$A \cap \bar{B}$} edge from parent node[left] {$P(A|\bar{B})$}}
			child{ node{$\bar{A} \cap \bar{B}$} edge from parent node[right] {$P(\bar{A}|\bar{B})$} }
		       }
	;

\end{tikzpicture}
% ****************************************************************

% ============================================================




% REFERENCES ============================================================
\cleardoublepage
%\renewcommand{\bibname}{Referenzen}	% Rename the bibliography title
\bibliographystyle{IEEEtran}	% Adds cites (Zitate)
\addcontentsline{toc}{section}{Referenzen}
% references can easily be generated using the OS X tool "BibDesk"
% Make sure you build your LaTeX document in BibTeX after defining your references
%	in order to make them valid.
\bibliography{references/book_ref1}
% ============================================================

% APPENDIX ============================================================
\cleardoublepage
\appendix
\section{Cheatsheet}
% ============================================================


\end{document}