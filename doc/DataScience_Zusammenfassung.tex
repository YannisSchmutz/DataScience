\documentclass{article}

% PACKAGES  ============================================================
\usepackage{lipsum}		% Used to generate dummy-text (\lipsum[1])
\usepackage[margin=2.54cm,includefoot]{geometry}		% Used to control margins
\usepackage{scrextend}	% To be able to use \begin{addmargin}

\usepackage{graphicx} % allows to import images
\usepackage{float}	% allows for control of float positions
\usepackage{tikz} 	% Used to create trees, neural networks, 
\usetikzlibrary{matrix,chains,positioning,decorations.pathreplacing,arrows} % For neural networks

\usepackage{pgfplots}	% to create plots

\usepackage{forest} % Used to create trees

\usepackage{amsmath}	 % to use cases in equations and vectors

\usepackage[hidelinks]{hyperref}	% allows for clickable references

\usepackage[numbers,sort&compress]{natbib} 	% sorts the cites in increasing order automatically when referenced and compresses successive references

\usepackage[utf8]{inputenc}		% Be able to use umlaute

\usepackage[ngerman]{babel}

\usepackage{fancyhdr}	% Used for Header and Footer stuff

\usepackage{xfrac}	% allows for slanted fractions 

\usepackage{pgfplots}	% Used for function-plots

\usepackage{amssymb} % Use commands like \mathbb
\usepackage{amsmath}
\usepackage{mathtools} % be able to use :=

\usepackage{algorithm}		% Wirte pseudocode algorithms
\usepackage{algorithmic}

%\usepackage[table]  % Used for tables

% ============================================================

% HEADER AND FOOTER STUFF ============================================================
\pagestyle{fancy}
\fancyhead{}	% clears header
\fancyfoot{}	% clears footer
\fancyfoot[R]{\thepage}	% sets position right
\renewcommand{\headrulewidth}{0pt}	% removes header line by setting it to zero
\renewcommand{\footrulewidth}{1pt}		% add footer line by setting it to one
% ============================================================

% DEFINE LIST ITEM BULLETS ============================================================
\renewcommand{\labelitemi}{$\bullet$}	% first list item
\renewcommand{\labelitemii}{$\circ$}	% one-indented item
\renewcommand{\labelitemiii}{$\diamond$}	% twice-indented item
% ============================================================

% BE ABLE TO USE ROW AND COLUMN VECTORS INLINE ============================================================
\newcommand{\icol}[1]{% inline column vector
  \left(\begin{smallmatrix}#1\end{smallmatrix}\right)%
}

\newcommand{\irow}[1]{% inline row vector
  \begin{smallmatrix}(#1)\end{smallmatrix}%
}
% ============================================================

% R Commands =================================================
% Be able to reference sections with number
\newcommand{\secref}[1]{\autoref{#1}. \nameref{#1}}
% ============================================================


% TODO: Diese beiden sollten jeweils das \begin{flushleft} erstzen. 
% Jedoch ist der Text dann im Blocksatz, wenn flushleft fehlt...
%Einrücken von Absätzen deaktivieren
\setlength{\parindent}{0pt}
%Zeilenabstand bei Abstätzen
\usepackage{parskip}

\begin{document}

% TITLE PAGE ============================================================
\begin{titlepage}
	
	\begin{center}
	\line(1,0){330} \\
	[2mm]
	\huge{\bfseries Data Science Zusammenfassung} \\
	[2mm]
	\line(1,0){320} \\
	[1,5cm]
	\textsc{\LARGE By Yannis Schmutz} \\
	[0.75cm]
	\textsc{\large todo} \\
	
	\end{center}
	
\end{titlepage}
% ============================================================

% PREFACE STUFF ============================================================
\pagenumbering{roman}		% sets the page numbering to roman for the preface etc.
\section*{Zusammenfassung}	% Adds a section without a number in front
\addcontentsline{toc}{section}{\numberline{}Zusammenfassung}	% adds a section without a number in front to the ToC
\cleardoublepage	% Finishes the current page so that the following page will always be odd.
% ****************************************************************

% TABLE OF CONTENTS ============================================================
\renewcommand{\contentsname}{Inhaltsverzeichnis}	% Rename table of contents to the german version
\tableofcontents		% adds table of contents (this needs to be compiled twice sometimes in order to update)
\thispagestyle{empty}	% removes header & footer on this page
\cleardoublepage	% Finishes the current page so that the following page will always be odd.
% ============================================================

% LIST OF FIGURES ============================================================
%\renewcommand{\listfigurename}{Abbildungsverzeichnis}	% renames list of figures
\listoffigures	% generates a list of figures
\addcontentsline{toc}{section}{Abbildungsverzeichnis}	% Adds list of figures to the ToC
\cleardoublepage
% ============================================================

% LIST OF TABLES ============================================================
%\renewcommand{\listtablename}{Tabellenverzeichnis}
\listoftables
\addcontentsline{toc}{section}{Tabellenverzeichnis}
\cleardoublepage
% ============================================================

% START OF REGULAR CHAPTERS ============================================================
\setcounter{page}{1}		% Sets this page to the first one (and not the table of contents)
\pagenumbering{arabic}	% Sets the page numbering back to arabic

\newpage
\section{Einleitung}
\label{sec:stat}


Dieses Dokument beschreibt diverse Themen im Bereich Data Engineering, Machine Learning und Data Science. Der Fokus liegt auf dem behandeltem Stoff des Moduls DENG2 der Berner Fachhochschule.

Prüfungsrelevant sind die folgenden Kapitel:

\begin{itemize}
  \item \secref{sec:weight_based_learning}
  \item \secref{sec:continuous_learning}
\end{itemize}

\include{chapters/2_Statistik}
\include{chapters/3_Probabilistik}
\include{chapters/4_FeatureScaling}
% ****************************************************************
\newpage
\section{Machine Learning}
\label{sec:ml}

\subsection{Übersicht}

\begin{figure}[H]
	\centering
	\label{fig:ml_overview}
	\begin{forest}
	for tree={draw, s sep=3em}
	[Machine Learning
	    [Überwachtes Lernen
	        [Klassifizierung
	            [Naive Bayes]
	             [k-N-N]
	             [NN]
	        ]
	        [Regression
	            [Linear Regression]
	            [NN]
	        ]
	    ]
	    [Unüberwachtes Lernen
	        [Clustering
	            [K-means]
	        ]
	        [Assoziierung]
	    ]
	]
	\end{forest}
	\caption{Machine Learning Kategorien}
\end{figure}

% #################################
\subsubsection{Überwachtes Lernen}
\begin{flushleft}

Überwachte Lern-Algorithmen probieren Beziehungen und Abhängigkeiten zwischen den Input-Features und des zu erzielenden Outputs zu erschliessen. Dies unter der Verwendung von \textbf{beschrifteten Daten}. Diese können zu Trainingszwecken verwendet werden. Jeder Satz an Daten besteht aus Input-Werten sowie einem dazugehörigen bekannten Output-Wert. Nach dem Trainieren des Algorithmus versucht dieser anhand von \textbf{neuen} Input-Features den dazugehörigen \textbf{unbekannten} Output vorherzusagen.
\linebreak

Das überwachte Lernen kann in zwei Kategorien unterteilt werden:

\begin{itemize}
	\item \textbf{Klassifizierung:} Ziel der Klassifizierung ist es, diskrete Werte vorherzusagen (bspw. Wahr/ Falsch, Spam-Mail/ normales Mail). 
	\item \textbf{Regression:} Das Ziel der Regression ist die Vorhersage kontinuierlicher Werte (bspw. Hauspreise in Abhängigkeit von Flaeche und Anzahl Zimmer).
\end{itemize}

\end{flushleft}

\subsubsection{Unueberwachtes Lernen}
\begin{flushleft}

Im unüberwachten Lernen stehen den Algorithmen \textbf{keine} beschrifteten Daten zur Verfügung. Die Algorithmen versuchen eigenständig Pattern in den zu behandelnden Daten zu erkennen und sie dadurch beispielsweise gruppieren zu können.
\end{flushleft}

% #################################

% ****************************************************************



\include{chapters/6_WeightBasedLearning}
\newpage
\section{Continuous Learning}
\label{sec:continuous_learning}

Beim obigen Beispiel (Perceptron) wurde der Error berechnet indem der Vorhersagewert vom erwarteten Wert subtrahiert wurde.

In diesem Kapitel werden Ansätze angeschaut, welche auch schauen \textbf{wie weit weg} die Vorhersage war.

\subsection{Adaline}
%\begin{flushleft}

Adaline steht für \textbf{Adaptive Linear Neuron} und ist ein \textbf{supervised classification Algorithmus}. Adaline funktioniert wie folgt:

\begin{itemize}
  \item Die Elemente des Input-Vektors (Featurevektors) werden jeweils mit einem Gewicht multipliziert und aufsummiert.
  \item Die Summe (z) wird einer \textbf{Aktivierungsfunktion} übergeben, welche den Wert $\hat{y}$ erzeugt.
  \item $\hat{y}$ wird dann verwendet, um den \textbf{Fehler} bzw. das \textbf{Update für die Gewichte} zu berechnen.
  \item $\hat{y}$ wird zudem einer \textbf{Schwellwertfunktion} übergeben, welche die Features einer Klasse zuordnet.
\end{itemize}




\newcommand{\myThresholdFunction}{
\draw[thick] %(-2.25em,0em) -- (1.25em,0em) 
			 (-0.5em,1.25em) -- (-0.5em,-1.25em)
(-0.5em,1.25em) -- (0.5em,1.25em)
(-0.5em,-1.25em) -- (-1.5em,-1.25em)
;}


\begin{figure}[H]
\centering
\label{fig:perceptron}
\begin{tikzpicture}[
     % define styles 
     clear/.style={ 
         draw=none,
         fill=none
     },
     net/.style={
         matrix of nodes,
         nodes={ draw, circle, inner sep=10pt },
         nodes in empty cells,
         column sep=1.5cm,
         row sep=-9pt
     },
     >=latex
]
% define matrix mat to hold nodes
% using net as default style for cells
\matrix[net] (mat)
{
% Define layer headings
|[clear]| \parbox{1.3cm}{\centering Input\\layer} & 
|[clear]| \parbox{1.3cm}{\centering Gewichtete\\Summe} &
|[clear]| \parbox{1.3cm}{\centering Aktivierungs\\funktion} &
|[clear]| \parbox{1.3cm}{\centering Schwellwert\\Funktion} \\
         
$+1$  		& |[clear]| & Error     & |[clear]| \\
|[clear]| 	& |[clear]| & |[clear]| & |[clear]| \\
$x_{1}$  	& |[clear]| & |[clear]| & |[clear]| \\
|[clear]| 	& $\Sigma$  & $\phi$ & \myThresholdFunction \\
\vdots  	& |[clear]| & |[clear]| & |[clear]| \\
|[clear]| 	& |[clear]| & |[clear]| & |[clear]| \\
$x_{n}$  	& |[clear]| & |[clear]| & |[clear]| \\
};
\draw[->] (mat-2-1) -- node[above=1mm] {$w_{0}$} (mat-5-2);
\draw[->] (mat-4-1) -- node[above=1mm] {$w_{1}$} (mat-5-2);
\draw[->] (mat-6-1) -- node[above=1mm] {$\vdots$} (mat-5-2);
\draw[->] (mat-8-1) -- node[above=1mm] {$w_{n}$} (mat-5-2);
\draw[->] (mat-5-2) -- node[above=1mm] {$z$} (mat-5-3);
\draw[->] (mat-5-3) -- node[above=1mm] {$\hat{y}$} (mat-5-4);
\draw[->] (mat-5-3) -- node[above=1mm] {$$} (mat-2-3);
\draw[->] (mat-2-3) -- node[above=1mm] {Update Gewichte} (-2.5cm, 1cm);

\draw[->] (mat-5-4) -- node[right=2em] {$\begin{cases}
       		1 &  \\
       		0 &
    	\end{cases}$} +(2cm,0);
\end{tikzpicture}
\caption{Adaline als Model}
\label{fig:adaline_model}
\end{figure}


% ============================================
\newpage
\subsubsection{Adaline vs Perceptron}

Ähnlich wie der Perceptron ist Adaline ein Einzellayer neuronales Netzwerk. Der Hauptunterschied liegt auf der Aktivierungsfunktion phi $\phi(z)$


\begin{itemize}
  \item Der \textbf{Perceptron} aktualisiert die Gewichte nur, wenn eine falsche Vorhersage getroffen wurde. Zudem wird die Error-Funktion erst nach der \textbf{Schwellwertfunktion} aufgerufen. Somit wird ihr stets immer nur eine 0 oder 1 übergeben.
  \item \textbf{Adaline} hingegen aktualisiert die Gewichte basierend auf einer \textbf{stetigen} Funktion (continuous). Der \textbf{Aktivierungsfunktion $\phi$}
 \end{itemize}



Wie in Abbildung \ref{fig:adaline_vs_perceptron} ersichtlich ist, werden die Gewichte bei Adaline \textbf{vor} der Entscheidungsfunktion aktualisiert.

\begin{figure}[h!]
	\includegraphics[scale=0.6]{figures/adaline_vs_perceptron}
	\caption{Adaline vs Perceptron}
	\label{fig:adaline_vs_perceptron}
\end{figure}

Im Gegensatz zum \textbf{binären Lernansatz} des Perceptron basieren viele supervised learning Algorithmen auf einer sogenannten \textbf{objective learning function}.


\newpage
\subsubsection{Objective Function}

Die Objective Function (Zielfunktion) ist mathematisch eine \textbf{Optimierung}.

Das Ziel hierbei ist es \textbf{optimale Parameter} zu finden. Optimal bedeutet den Output der Funktion entweder zu \textbf{maximieren} oder zu \textbf{minimieren}. 


Bei Machine Learning entsprechen die Parameter den \textbf{Gewichten}.


Eine Objective Funktion berechnet einen Output basierend auf den Eingaben:

\begin{itemize}
  \item Vorhersage (prediction)
  \item Eigentlicher Wert (labelled value)
\end{itemize}

Hierbei soll deren Differenz möglichst klein werden. Somit ist die Vorhersage sehr nahe am eigentlichen Wert. Ein Beispiel dafür zeigt Abbildung \ref{fig:objective_minimum}. 


\begin{figure}[h!]
	\includegraphics[scale=0.4]{figures/objective_minimum}
	\caption{Beispiel Zielfunktion}
	\label{fig:objective_minimum}
\end{figure}



Objective Funktion ist ein sehr genereller Term im Bereich ML.
Meistens wollen wir den Output der Objective Funktion \textbf{minimieren}. In diesem Fall sprechen wir von einer \textbf{Cost Function} oder \textbf{Loss Function}.


Wollen wir den Output \textbf{maximieren} so sprechen wir von einer \textbf{Likelihood Maximization} Funktion.



 
\newpage
\subsubsection{Objective Function in Adaline}

Adaline verwendet die Cost Function \textbf{Sum of Squared Errors (SSE)}.

SSE summiert alle quadrierten Differenzen zwischen Vorhersage und effektivem Wert auf.

$$ SSE = \frac{1}{2} \sum_{i=1}^{m}(y^{(i)} - \hat{y}^{(i)})^{2} $$

Hier eine detailliertere Darstellung von $\hat{y}$

$$ SSE = \frac{1}{2} \sum_{i=1}^{m}(y^{(i)} - \phi(z^{(i)}))^{2} $$


Wichtig zu wissen:
\begin{itemize}
  \item Die Funktion SSE ist \textbf{differenzierbar (ableitbar)}. Daher kann für jeden Punkt die Steigung berechnet werden.
  \item Sie hat ein \textbf{globales Minimum}.
\end{itemize}


Diese beiden Punkte sind notwendig für \textbf{Optimierungsalgorithmen}. Diese helfen dabei den Wert der Gewichte zu bestimmen damit die Cost Function möglichst klein wird. (Bsp. Gradient Descent)


\subsubsection{Adaline Learning Rule}

$$ \Delta w_{j} = \eta \sum_{i=1}^{m} (y^{(i)} - \phi(z^{(i)})) * x_{j}^{(i)} $$


\begin{align*}
	\Delta w_{j}  &= \text{Gewichtsupdate} \\
	\eta &= \text{Learning Rate} \\
	m &= \text{Anzahl Samples}	\\
	y^{(i)} &= \text{Effektiver Zielwert (Label)} \\
	\phi &= \\
	z^{(i)} &= \\
	x_{j}^{(i)} &= \text{Feature j des Samples i} \\
\end{align*}


$$ w := w + \Delta w$$











TODO:
This update rule is in fact the stochastic gradient descent update for linear regression.[7]






\include{chapters/GradientDescent}

\include{chapters/NormalEquation}
\newpage
\section{Linear Regression}
\begin{flushleft}

Die lineare Regression ist ein Verfahren welches versucht, eine abhängige Variable durch eine oder mehrere unabhängige Variablen zu erklären.

\subsection{Simple lineare Regression}

Die simple lineare Regression arbeitet lediglich im zweidimensionalen Raum.

\begin{tikzpicture}
\begin{axis} [
	axis lines = left, % Only displays axis on left and bottom (not whole box)
	ymin = 0,	% y-axis shall always start at zero
	xlabel = $x$,
	ylabel = $f(x)$,
	mark=*,
	scatter/classes={%	Defines the point appearance for the scatter plot
		a={blue}%,
		%b={mark=triangle*,red},
		%c={mark=o,draw=black}
		}
	]

\addplot [
	domain = 0:60, 	% Range for the value x
	samples = 100,	% Determines the number of points in the interval defined by domain. 
	color = red,	% Color of the
	] {0.83*x + 10.44};
\addlegendentry {$mx + q$}

\addplot [only marks, scatter, scatter src=explicit symbolic]
	table [meta=class] {		% meta defines the column to use for the class of the point
		x		y		class
		10		20		a
		10		15		a
		15		17		a
		20		21		a
		20		31		a
		25		40		a
		30		27		a
		35		40		a
		30		55		a
		40		45		a
		40		37		a
		45		52		a
		50		55		a
		50		45		a
		55		50		a
		60		65		a
		
	};

\end{axis}
\end{tikzpicture}

Die Steigung [m] sowie der y-Achsenabschnitt [q] lassen sich wie folgt berechnen:

$$m = \dfrac{\sum_{i=1}^n (x_{i} - \bar{x})(y_{i} - \bar{y})}
                       {\sum_{i=1}^n (x_{i} - \bar{x})^{2}}$$

$$q = \bar{y} - m\bar{x}$$

Wobei die werte $\bar{x}$, $\bar{y}$ den arithmetischen Mitteln der Definitions- und Bildmenge entsprechen.
\linebreak
Die Qualität eines Regressionsmodells kann durch Genauigkeitsmetriken bestimmt werden. Hierbei werden stets die effektiven Y-Werte mit den jeweiligen Werten $\hat{y}_{i} = mx_{i} + q$ der linearen Regressionslinie verglichen.


Der \textbf{Mittlerer absoluter Fehler} (mean-absolute-error) ist die simpleste Metrik und zeig den effektiven Durchschnittsfehelr auf. 
$$MAE = \dfrac{1}{n}\sum_{i=1}^n|y_{i} - \hat{y}_{i}|$$


Die \textbf{Mittlere quadratische Abweichung} (mean-square-error) reagiert aufgrund des Exponenten proportional stärker auf grosse Fehler.
$$MSE = \dfrac{1}{n}\sum_{i=1}^n(y_{i} - \hat{y}_{i})^{2}$$

Der \textbf{root-mean-square-error} ist die gängigste Metrik, da dieser, durch das Ziehen der Wurzel, in der gleichen Einheit interpretierbar ist wie die eigentlichen Y-Vektoren.
$$RMSE = \sqrt{\dfrac{1}{n}\sum_{i=1}^n(y_{i} - \hat{y}_{i})^{2}} = \sqrt{MSE}$$


\subsection{Polynomial Regression}
\subsection{Multivariable Regression}

\end{flushleft}
\newpage
\section{Logistic Regression}
\begin{flushleft}


Mittels Logistischer Regression lassen sich diskrete Phänomene klassifizieren. Die Funktion des Models ist wie folgt definiert:


$$ h_{\Theta}(x) = g(\Theta^{T}x) $$ 
$$ g(z) = \frac{1}{1 + e^{-z}} $$
$$ h_{\Theta}(x) = \frac{1}{1 + e^{-\Theta^{T}x}} $$

Die Funktion $g(x)$ ist hierbei die sogenannte Sigmoid (oder auch logistische) Funktion.

\begin{tikzpicture}[declare function={sigma(\x)=1/(1+exp(-\x));}]
\begin{axis}%
[
    grid=major,     
    xmin=-8,
    xmax=8,
    axis x line=bottom,
    ytick={0,.5,1},
    ymax=1,
    axis y line=middle,
    samples=100,
    domain=-8:8,
    legend style={at={(1,0.9)}}     
]
    \addplot[blue,mark=none]   (x,{sigma(x)});
    \legend{$g(x)$}
\end{axis}
\end{tikzpicture}

Der Wert $h_{\Theta}(x)$ wird als Wahrscheinlichkeit verstanden, dass der Output (y) positive ist für ein gegebener Eingabewert (x) parametrisiert mit $\Theta$.
\linebreak

Formal definiert:
$$h_{\Theta}(x) = P(y=1|x;\Theta)$$
Somit gilt:
$$ P(y=0|x;\Theta) = 1 - P(y=1|x;\Theta)$$


\end{flushleft}




\newpage
\section{Neuronale Netzwerke}
\subsection{Representation}
\begin{flushleft}


Der Inputnode $x_{0}$ wird nicht immer eingezeichnet und repräsentiert den Bias-Node.

Weiter gilt:

$\begin{aligned}
 \alpha_{i}^{(j)} &= \text{Aktivierung der Einheit i im Layer j}
\end{aligned}$

$\begin{aligned}
 \Theta^{(j)} &= \text{Gewichtsmatrix welche Layer j auf Layer j+1 mapt}
\end{aligned}$

\begin{tikzpicture}[
     % define styles 
     clear/.style={ 
         draw=none,
         fill=none
     },
     net/.style={
         matrix of nodes,
         nodes={ draw, circle, inner sep=10pt },
         nodes in empty cells,
         column sep=2cm,
         row sep=-9pt
     },
     >=latex
]
% define matrix mat to hold nodes
% using net as default style for cells
\matrix[net] (mat)
{
% Define layer headings
|[clear]| \parbox{1.3cm}{\centering Input\\layer} 
    & |[clear]| \parbox{1.3cm}{\centering Hidden\\layer} 
    & |[clear]| \parbox{1.3cm}{\centering Output\\layer} \\
         
$x_{0}$  & |[clear]|        & |[clear]| \\
|[clear]|         & $\alpha_{1}^{(2)}$ & |[clear]| \\
$x_{1}$  & |[clear]|        & |[clear]| \\
|[clear]|         & |[clear]|        & |[clear]| \phantom{$a_{0}^{0}$} \\
$x_{2}$  & $\alpha_{2}^{(2)}$ & $$ \\
|[clear]|         & |[clear]|        & |[clear]|  \phantom{$a_{0}^{0}$} \\
$x_{3}$  & |[clear]|        & |[clear]| \\
|[clear]|         & $\alpha_{3}^{(2)}$ & |[clear]| \\
$x_{4}$  & |[clear]|        & |[clear]| \\ 
};
% left most lines into input layers
\foreach \ai in {2,4,6,8,10}
    \draw[<-] (mat-\ai-1) -- +(-2cm,0);
% lines from a_{i}^{0} to each a_{j}^{1}
\foreach \ai in {2,4,6,8,10} {
    \foreach \aii in {3,6,9}
        \draw[->] (mat-\ai-1) -- (mat-\aii-2);
        }
% lines from a_{i}^{1} to a_{0}^{2}
\foreach \ai in {3,6,9}
  \draw[->] (mat-\ai-2) -- (mat-6-3);
    
% right most line with Output label
\draw[->] (mat-6-3) -- node[above] {$h_{\Theta}(x)$} +(2cm,0);
\end{tikzpicture}




$$ \alpha_{1}^{(2)} = g(\Theta_{10}^{(1)}x_{0} +  \Theta_{11}^{(1)}x_{1} + \Theta_{12}^{(1)}x_{2} +\Theta_{13}^{(1)}x_{3} + \Theta_{14}^{(1)}x_{4})$$

$$ \alpha_{2}^{(2)} = g(\Theta_{20}^{(1)}x_{0} +  \Theta_{21}^{(1)}x_{1} + \Theta_{22}^{(1)}x_{2} +\Theta_{23}^{(1)}x_{3} + \Theta_{24}^{(1)}x_{4})$$

$$ \alpha_{3}^{(2)} = g(\Theta_{30}^{(1)}x_{0} +  \Theta_{31}^{(1)}x_{1} + \Theta_{32}^{(1)}x_{2} +\Theta_{33}^{(1)}x_{3} + \Theta_{34}^{(1)}x_{4})$$


$$ h_{\Theta}(x) = \alpha_{1}^{(3)} = g(\Theta_{10}^{(2)}\alpha_{0}^{(2)} + \Theta_{11}^{(2)}\alpha_{1}^{(2)} + \Theta_{12}^{(2)}\alpha_{2}^{(2)} + \Theta_{13}^{(2)}\alpha_{3}^{(2)}) $$


Existiert ein neuronales Netz mit $s_{j}$ Einheiten im Layer $j$, $s_{j+1}$ Einheiten im Layer $j+1$, dann hat $\Theta^{(j)}$ die Dimension $s_{j+1} \times (s_{j} + 1)$.

\end{flushleft}


\subsection{Logik Beispiel}

\subsubsection{AND}
\begin{flushleft}


Ein AND-Gatter kann wie folgt erstellt werden:

\begin{tikzpicture}[
     % define styles 
     clear/.style={ 
         draw=none,
         fill=none
     },
     net/.style={
         matrix of nodes,
         nodes={ draw, circle, inner sep=10pt },
         nodes in empty cells,
         column sep=2cm,
         row sep=-9pt
     },
     >=latex
]
% define matrix mat to hold nodes
% using net as default style for cells
\matrix[net] (mat)
{
% Define layer headings
|[clear]| \parbox{1.3cm}{\centering Input\\layer} & |[clear]| \parbox{1.3cm}{\centering Output\\layer} \\
         
$+1$  		& |[clear]| \\
|[clear]| 	& |[clear]| \\
$x_{1}$  	& |[clear]| \\
|[clear]| 	& $$ \\
$x_{2}$  	& |[clear]| \\
};
\draw[->] (mat-2-1) -- node[above=1mm] {-30} (mat-5-2);
\draw[->] (mat-4-1) -- node[above=1mm] {20} (mat-5-2);
\draw[->] (mat-6-1) -- node[above=1mm] {20} (mat-5-2);
\draw[->] (mat-5-2) -- node[above] {$h_{\Theta}(x)$} +(2cm,0);
\end{tikzpicture}

$$ h_{\Theta}(x) = g(-30 + 20x_{1} + 20x_{2}) $$


\begin{center}
\begin{tabular}{ c c|r } 

 $x_{1}$ & $x_{2}$ & $h_{\Theta}(x)$ \\ 
  \hline
 0 & 0 & $g(-30) \approx 0$ \\ 
 0 & 1 & $g(-10) \approx 0$ \\ 
 1 & 0 & $g(-10) \approx 0$ \\ 
 1 & 1 & $g(10) \approx 1$ \\ 

\end{tabular}
\end{center}
\end{flushleft}


\subsubsection{OR}
\begin{flushleft}


Ein OR-Gatter kann wie folgt erstellt werden:

\begin{tikzpicture}[
     % define styles 
     clear/.style={ 
         draw=none,
         fill=none
     },
     net/.style={
         matrix of nodes,
         nodes={ draw, circle, inner sep=10pt },
         nodes in empty cells,
         column sep=2cm,
         row sep=-9pt
     },
     >=latex
]
% define matrix mat to hold nodes
% using net as default style for cells
\matrix[net] (mat)
{
% Define layer headings
|[clear]| \parbox{1.3cm}{\centering Input\\layer} & |[clear]| \parbox{1.3cm}{\centering Output\\layer} \\
         
$+1$  		& |[clear]| \\
|[clear]| 	& |[clear]| \\
$x_{1}$  	& |[clear]| \\
|[clear]| 	& $$ \\
$x_{2}$  	& |[clear]| \\
};
\draw[->] (mat-2-1) -- node[above=1mm] {-10} (mat-5-2);
\draw[->] (mat-4-1) -- node[above=1mm] {20} (mat-5-2);
\draw[->] (mat-6-1) -- node[above=1mm] {20} (mat-5-2);
\draw[->] (mat-5-2) -- node[above] {$h_{\Theta}(x)$} +(2cm,0);
\end{tikzpicture}

$$ h_{\Theta}(x) = g(-10 + 20x_{1} + 20x_{2}) $$


\begin{center}
\begin{tabular}{ c c|r } 

 $x_{1}$ & $x_{2}$ & $h_{\Theta}(x)$ \\ 
  \hline
 0 & 0 & $g(-10) \approx 0$ \\ 
 0 & 1 & $g(10) \approx 1$ \\ 
 1 & 0 & $g(10) \approx 1$ \\ 
 1 & 1 & $g(30) \approx 1$ \\ 

\end{tabular}
\end{center}
\end{flushleft}


\subsubsection{NOT}
\begin{flushleft}


Ein NOT-Gatter kann wie folgt erstellt werden:

\begin{tikzpicture}[
     % define styles 
     clear/.style={ 
         draw=none,
         fill=none
     },
     net/.style={
         matrix of nodes,
         nodes={ draw, circle, inner sep=10pt },
         nodes in empty cells,
         column sep=2cm,
         row sep=-9pt
     },
     >=latex
]
% define matrix mat to hold nodes
% using net as default style for cells
\matrix[net] (mat)
{
% Define layer headings
|[clear]| \parbox{1.3cm}{\centering Input\\layer} & |[clear]| \parbox{1.3cm}{\centering Output\\layer} \\
         
$+1$  		& |[clear]| \\
|[clear]| 	& $$ \\
$x_{1}$  	& |[clear]| \\
};
\draw[->] (mat-2-1) -- node[above=1mm] {10} (mat-3-2);
\draw[->] (mat-4-1) -- node[above=1mm] {-20} (mat-3-2);
\draw[->] (mat-3-2) -- node[above] {$h_{\Theta}(x)$} +(2cm,0);
\end{tikzpicture}

$$ h_{\Theta}(x) = g(10 - 20x_{1}) $$


\begin{center}
\begin{tabular}{ c|r } 

 $x_{1}$ & $h_{\Theta}(x)$ \\ 
  \hline
 0 & $g(10) \approx 1$ \\ 
 1 & $g(-10) \approx 0$ \\ 

\end{tabular}
\end{center}
\end{flushleft}



\subsubsection{XNOR}
\begin{flushleft}


Durch Kombinationen von einzelnen NN können komplexere Gatter erstellt werden. Wie bspw. das XNOR Gatter.

\begin{tikzpicture}[
     % define styles 
     clear/.style={ 
         draw=none,
         fill=none
     },
     net/.style={
         matrix of nodes,
         nodes={ draw, circle, inner sep=10pt },
         nodes in empty cells,
         column sep=2cm,
         row sep=-9pt
     },
     >=latex
]
% define matrix mat to hold nodes
% using net as default style for cells
\matrix[net] (mat)
{
% Define layer headings
|[clear]| \parbox{1.3cm}{\centering Input\\layer} 
	& |[clear]| \parbox{1.3cm}{\centering Hidden\\layer} 
	& |[clear]| \parbox{1.3cm}{\centering Output\\layer} \\
         
$+1$  		& $+1$ 					&	|[clear]|\\
|[clear]|	& |[clear]|				&	|[clear]| \\
$x_{1}$  	& $\alpha_{1}^{(2)}$  	&  	$\alpha_{1}^{(3)}$\\
|[clear]|	& |[clear]|				&	|[clear]| \\
$x_{2}$  	& $\alpha_{1}^{(2)}$ 	&  	|[clear]|\\
};
% +1
\draw[->] (mat-2-1) -- node[above=0mm] {-30} (mat-4-2);
\draw[->] (mat-2-1) -- node[above=0mm] {10} (mat-6-2);
%  x1
\draw[->] (mat-4-1) -- node[below=0mm] {20} (mat-4-2);
\draw[->] (mat-4-1) -- node[above=0mm] {-20} (mat-6-2);
%  x2
\draw[->] (mat-6-1) -- node[below=0mm] {20} (mat-4-2);
\draw[->] (mat-6-1) -- node[below=0mm] {-10} (mat-6-2);

%
\draw[->] (mat-2-2) -- node[above=0mm] {-10} (mat-4-3);
\draw[->] (mat-4-2) -- node[above=0mm] {20} (mat-4-3);
\draw[->] (mat-6-2) -- node[above=0mm] {20} (mat-4-3);

\draw[->] (mat-4-3) -- node[above] {$h_{\Theta}(x)$} +(2cm,0);

\end{tikzpicture}


\begin{center}
\begin{tabular}{ c c|c c|r } 

 $x_{1}$ & $x_{2}$ & $\alpha_{1}^{(2)}$ & $\alpha_{2}^{(2)}$ & $h_{\Theta}(x)$ \\ 
  \hline
 0 & 0 & 0 & 1 & 1 \\ 
 0 & 1 & 0 & 0 & 0 \\ 
 1 & 0 & 0 & 0 & 0 \\ 
 1 & 1 & 1 & 0 & 1 \\ 

\end{tabular}
\end{center}
\end{flushleft}





% ============================================================



% ============================================================


%\newpage
\section{Tryout}\label{sec:tryout}

\begin{figure}[H]	% H stands for here (place right here)
	\centering
	\includegraphics[height=5cm]{figures/tryout.png}
	\caption[Optional optional]{Entscheidungsbaum}
	\label{fig:tryout}
\end{figure}

Wie auf der Abbildung \ref{fig:tryout} zu sehen ist.....


\begin{table}[H]
	\centering
	\label{tab:tryouttab}
\caption[This is an optional caption, without reference]{Local caption, with reference}
	\cite{ref:ds_1, ref:nn_1, ref:ai_1}	% Used to add cites (zitieren)

	\begin{tabular}{l c r}
		Area & Number of rooms & Price \\ \hline
		80	& 4				& 1680 \\
		100	& 5				& 2300 \\
		50	& 2.5				& 1500 \\

	\end{tabular}
\end{table}


\begin{itemize}
	\item This is an item
	\item This is another item
	\begin{itemize}
		\item This is a further item
		\item [blub] This is an item with a custom bullet point
	\end{itemize}
\end{itemize}

\begin{enumerate}
	\item This is a numbered item
	\item And so on
\end{enumerate}


\newpage
\subsection{Math examples}

Here's an example within a sentence $E =mc^2$.

And here one example $$a=v/t$$ which is centred. \\

$$-\frac{\hbar^2}{2m}\frac{d^2\Psi}{dx^2} = E\Psi$$

Fractions

$$d = v_it + \frac{1}{2} \cdot at^2$$
$$d = v_it + \sfrac{1}{2} \cdot at^2$$


Brackets:
$$\left( \frac{1}{2} \right) \cdot 2 = 1$$	% use \left( ..... \right) to match the brackets to the content
$$\left| -7 \right| = 7$$
$$\sqrt{4} = 2$$
$$\sqrt{4} \ne 1$$
$$\sqrt{4} < 5$$
$$ \pi \approx 3 $$
$$ \pi \times \sqrt{4} < 15 $$

\begin{eqnarray}	% Equation array
	3x + 14 &=& 20 \\
	3x &=& 6 \\
	x &=& 2
\end{eqnarray}

\begin{equation}
\label{eq:first}
x^2 + 3x - 7 = 0
\end{equation}

\newpage
\subsection{Graphs}




\begin{tikzpicture}[sibling distance=12em,
							%root/.style={treenode,circle,draw},
							every child node/.style={circle, draw=black},
							]
%	[align=center, sibling distance=5cm]
	\node[fill=black]{}
		child { node {B}
		[sibling distance=6em]
			child { node{A $\cap$ B}  edge from parent node[left] {$P(A|B)$} }
			child { node{$\bar{A} \cap B$} edge from parent node[right] {$P(\bar{A}|B)$}}
			}
		child{ node{$\bar{B}$}
			[sibling distance=6em]
			child{ node{$A \cap \bar{B}$} edge from parent node[left] {$P(A|\bar{B})$}}
			child{ node{$\bar{A} \cap \bar{B}$} edge from parent node[right] {$P(\bar{A}|\bar{B})$} }
		       }
	;

\end{tikzpicture}
% ****************************************************************

% ============================================================




% REFERENCES ============================================================
\cleardoublepage
%\renewcommand{\bibname}{Referenzen}	% Rename the bibliography title
\bibliographystyle{IEEEtran}	% Adds cites (Zitate)
\addcontentsline{toc}{section}{Referenzen}
% references can easily be generated using the OS X tool "BibDesk"
% Make sure you build your LaTeX document in BibTeX after defining your references
%	in order to make them valid.
\bibliography{references/book_ref1}
% ============================================================

% APPENDIX ============================================================
\cleardoublepage
\appendix
\section{Cheatsheet}
% ============================================================


\end{document}